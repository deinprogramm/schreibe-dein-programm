\documentclass{tueteubner}

\usepackage[latin1]{inputenc}
\pagestyle{empty}

\begin{document}

\sffamily

\centerline{\Large Michael Sperber \qquad Herbert Klaeren}

\medskip

\centerline{\textbf{\huge Die Macht der Abstraktion}}

\medskip

\centerline{\Large Einführung in die Programmierung}

\medskip

\textit{Die Macht der Abstraktion} ist eine Einführung in die
Entwicklung von Programmen und die dazugehörigen formalen Grundlagen.
Im Zentrum stehen \textit{Konstruktionsanleitungen}, welche die
systematische Konstruktion von Programmen fördern, sowie Techniken zur
\textit{Abstraktion}, welche die Umsetzung der
Konstruktionsanleitungen ermöglichen.  In der Betonung systematischer
Konstruktion unterscheidet sich dieses Buch drastisch von den meisten
anderen Einführungen in die Programmierung.

Die vermittelten Grundlagen und Techniken sind unabhängig von einer
bestimmten Programmiersprache.  Zur Illustration und zum Training der
Programmierung dient \textit{Scheme}, eine kleine und leicht
erlernbare Programmiersprache, die es erlaubt, die Konzepte der
Programmierung zu präsentieren, ohne Zeit mit der Konstruktvielfalt
anderer Programmiersprachen zu verlieren.  Entsprechend vermittelt
dieses Buch fortgeschrittene Techniken.  Scheme-Könner sind in der
Lage, andere Programmiersprachen in kürzester Zeit zu erlernen.

\textit{Die Macht der Abstraktion} ist aus der Praxis der
Informatik-Grundausbildung an der Universität Tübingen entstanden:
Über mehrere Vorlesungszyklen wurden Stoffauswahl und Präsentation
stetig verbessert.  Gegenüber dem Vorgängerbuch \textit{Vom
  Problem zum Programm} wurde ein Großteil des Materials neu
entwickelt.  Das Buch enthält viele
Beispiele und Übungsaufgaben.  Alle nötigen mathematischen
Grundlagen werden vermittelt.

\subsubsection*{Inhalt}~

\vspace*{-0.5ex}

\setlength\itemsep{-4pt}
\medskip

\begin{minipage}[t]{0.5\textwidth}
\begin{itemize}
\item Was ist Informatik?
\item Elemente des Programmierens
\item Fallunterscheidungen und Verzweigungen
\item Zusammengesetzte und gemischte Daten
\item Induktive Definitionen
\item Rekursion
\item Praktische Programme mit Listen
\item Higher-Order-Programmierung
\end{itemize}
\end{minipage}
\qquad
\begin{minipage}[t]{0.5\textwidth}
\begin{itemize}
\item Zeitabhängige Modelle
\item Abstrakte Datentypen
\item Bin\"are B\"aume
\item Zuweisungen und Zustand
\item Objektorientiertes Programmieren
\item Logische Kalk\"ule
\item Der $\lambda $-Kalk\"ul
\item Interpretation von Scheme
\end{itemize}
\end{minipage}

\subsubsection*{Zielgruppen}~

\vspace*{-2ex}

\begin{itemize}
\item Studierende der Informatik im Haupt- und Nebenfach an
  Fachhochschulen und Universitäten
\item Studierende, die einen spannenden Einstieg in die Programmierung
  suchen
\end{itemize}

\vspace{-2.5ex}

\subsubsection*{Die Autoren}~

\vspace*{-0.5ex}

Prof.~Dr.~Herbert Klaeren, Universität Tübingen\\
Dr.~Michael Sperber, freiberuflicher Software-Entwickler

\vfill

\end{document}
