% $Id$
\chapter{Weitere Programmelemente}
\label{cha:elements}

\section{Zwischenergebnisse}

\begin{feature}{\texttt{let*}}{scheme:letast}
\texttt{Let*}\index{let*@\texttt{let*}} ist eine sequentielle
Variante von \texttt{let} mit der gleichen äußeren Form:
%
\begin{alltt}
(let* ((\(v\sb{1}\) \(e\sb{1}\)) \(\ldots\) (\(v\sb{n}\) \(e\sb{n}\))) \(b\))
\end{alltt}
%
Während bei \texttt{let} keine der $v_i$ in den $e_i$ sichtbar ist,
sind bei \texttt{let*} in jedem $e_i$ die \emph{vorhergehenden} $v_j$
für $j<i$ sichtbar.  Es gilt also:
%
\begin{verbatim}
(define x 23)
> (let* ((x 42)
         (y (- x 19)))
    (+ x y))
65
\end{verbatim}
%
\texttt{Let*} ist ebenfalls syntaktischer Zucker und wird in eine
Serie geschachtelter \texttt{let}-Ausdrücke übersetzt:
%
\begin{alltt}
(let* ((\(v\sb{1}\) \(e\sb{1}\)) (\(v\sb{2}\) \(e\sb{2}\)) \(\ldots\)) \(b\))
\(\Longrightarrow\) (let ((\(v\sb{1}\) \(e\sb{1}\))) (let* ((\(v\sb{2}\) \(e\sb{2}\)) \(\ldots\)) \(b\)))
\end{alltt}
\end{feature}
%
\begin{verbatim}
(define x 23)
(let ((x 42)
      (y (- x 19)))
  (+ x y))
\end{verbatim}
%
Allerdings gibt es auch eine sequentielle Form von \texttt{let}~---
\texttt{let*}~--- die in Abbildung~\ref{scheme:letast} erläutert ist.
\section*{Aufgaben}

\begin{aufgabe}
\begin{enumerate}\item
  Schreibe eine Funktion \texttt{min} mit zwei Parametern, welche als
  Ergebnis die kleinere zweier Zahlen liefert.  Das Gerüst könnte so
  aussehen:
  %
\begin{verbatim}
(define min
  (lambda (number-1 number-2)
    ...))
\end{verbatim}
\item Schreibe eine Funktion namens \texttt{min-3} mit \emph{drei}
  Parametern, die deren Minimum berechnet, ohne dass im Rumpf von
  \texttt{min-3} ein \texttt{if} vorkommt.
\item
  Schreibe eine Funktion, die drei Zahlen als Argumente nimmt, und die
  Summe der Quadrate der beiden größeren Zahlen zurückgibt.
\end{enumerate}
\end{aufgabe}

\begin{aufgabe}
Betrachte die folgende Definition:
%
\begin{verbatim}
(define foo
  (lambda (bar)
    (let* ((baz (- bar 1))
           (bar baz))
      ((lambda (bar)
         (if (= baz 0)
             1
             (* bar (foo baz))))
       (* 2 bar)))))
\end{verbatim}
    %
  \begin{enumerate}
  \item Welchen Wert hat der Ausdruck \texttt{(foo 3)}?
  \item Führe die Auswertung von \texttt{(foo 1)} nach dem
    Substitutionsmodell durch!
  \end{enumerate}
\end{aufgabe}

%%% Local Variables: 
%%% mode: latex
%%% TeX-master: "i1"
%%% End: 
