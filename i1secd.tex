% Diese Datei ist Teil des Buchs "Schreibe Dein Programm!"
% Das Buch ist lizensiert unter der Creative-Commons-Lizenz
% "Namensnennung - Weitergabe unter gleichen Bedingungen 4.0 International (CC BY-SA 4.0)"
% https://creativecommons.org/licenses/by-sa/4.0/deed.de

\chapter{Die SECD-Maschine}\label{cha:secd}

Der $\lambda$-Kalkül ist als theoretisches Modell für berechenbare
Funktionen lange vor der Erfindung des Computers entwickelt worden.
Die Reduktionsregeln dienen dabei der Entwicklung von Beweisen über
die Äquivalenz von $\lambda$-Termen.  Damit der $\lambda$-Kalkül auch
als Modell für die tatsächliche Ausführung von Programmen, auf dem
Computer geeignet ist, fehlen noch zwei Zutaten: die direkte
Definition von "<eingebauten"> Werten und Operationen wie Zahlen und
booleschen Werten sowie ein formales Auswertungsmodell.  Dieses
Kapitel stellt zunächst den \textit{angewandten $\lambda$-Kalkül} vor,
der den normalen $\lambda$-Kalkül um primitive Werte und Operationen
erweitert, und dann die \textit{SECD-Maschine}, ein klassisches
Auswertungsmodell für die Call-by-Value-Reduktion.  Angenehmerweise
lässt sich die SECD-Maschine auch als Programm implementieren,
was ebenfalls in diesem Kapitel geschieht.  Die SECD-Maschine
kennt keine Zuweisungen; es folgt darum noch die Darstellung der
\textit{SECDH-Maschine}, die auch einen \textit{Speicher} kennt und
damit Zuweisungen korrekt modelliert.

Ab diesem Kapitel wird die Sprachebene \texttt{Schreibe
  Dein Programm! - fortgeschritten} benötigt. 

\section{Der angewandte $\lambda$-Kalkül}

Abschnitt~\ref{sec:lambdaprog} zeigte bereits, dass sich auch boolesche
Werte und Zahlen im $\lambda$-Kalkül durch $\lambda$-Terme darstellen
lassen.  Das ist zwar aus theoretischer Sicht gut zu wissen, auf Dauer
aber etwas mühsam: Darum ist es sinnvoll, mit einer erweiterten
Version des $\lambda$-Kalküls zu arbeiten, die solche "<primitiven">
Werte direkt kennt.  Abschnitt~\ref{sec:lambdaprog} hat gezeigt,
dass eine solche Erweiterung nur syntaktischer Zucker ist,
also die Ausdruckskraft des Kalküls nicht wirklich erhöht.  Alle
Erkenntnisse aus dem normalen $\lambda$-Kalkül bleiben also erhalten.

Ein solcher erweiterter $\lambda$-Kalkül heißt auch
\textit{angewandter $\lambda$-Kalkül}:

\begin{definition}[Sprache des angewandten $\lambda$-Kalküls
  $\mathcal{L}_{\lambda{}A}$]\index{angewandter $\lambda$-Kalkül}\label{def:lambda-angewandt}
  
  Sei $V$ eine abzählbare Menge von Variablen.  Sei $B$ eine Menge von
  \textit{Basiswerten\index{Basiswert}}.  Sei für eine natürliche
  Zahl $n$ und $i \in \{1, \ldots, n\}$ jeweils $\Sigma^i$ eine Menge
  von \textit{$i$-stelligen Primitiva\index{Primitivum}}~-- die Namen
  von "<eingebauten Operationen">.  Jedem $F^i\in\Sigma^i$ ist eine
  $i$-stellige Funktion $F_B^i: B\times\ldots\times B \rightarrow
  B$~-- ihre \textit{Operation}~--
  zugordnet.  Die Sprache des angewandten $\lambda$"=Kalküls, die
  Menge der \textit{angewandten $\lambda$-Terme},
  $\mathcal{L}_{\lambda{}A}$\index{L@$\mathcal{L}_{\lambda{}A}$}, ist
  durch folgende Grammatik definiert:
  \begin{grammar}
    \meta{$\mathcal{L}_{\lambda{}A}$} \: \meta{$V$}
    \> \| (\meta{$\mathcal{L}_{\lambda{}A}$} \meta{$\mathcal{L}_{\lambda{}A}$})
    \> \| ($\lambda$\meta{$V$}.\meta{$\mathcal{L}_{\lambda{}A}$})
    \> \| \meta{$B$}
    \> \| (\meta{$\Sigma^1$}~\meta{$\mathcal{L}_{\lambda{}A}$})
    \> \| (\meta{$\Sigma^2$}~\meta{$\mathcal{L}_{\lambda{}A}$}~\meta{$\mathcal{L}_{\lambda{}A}$})
    \> \ldots
    \> \| (\meta{$\Sigma^n$}~\meta{$\mathcal{L}_{\lambda{}A}$}~\ldots~\meta{$\mathcal{L}_{\lambda{}A}$}) \quad \textrm{($n$-mal)}
  \end{grammar}
  Die Grammatik ist abgekürzt notiert: Die letzen Klauseln besagen,
  dass es für jede Stelligkeit $i$ eine Klausel mit \meta{$\Sigma^i$} gibt,
  bei der jeweils $i$ Wiederholungen von
  \meta{$\mathcal{L}_{\lambda{}A}$}~-- entsprechend der Stelligkeit
  der Primitiva in $\Sigma^i$.
  %
Dabei heißen Terme der Form $(F^k~e_1~\ldots~e_k)$ auch
\textit{primitive Applikationen\index{primitive Applikation}}.
\end{definition}
%
In diesem Kapitel dienen normalerweise die Zahlen als Basiswerte mit
den üblichen Operationen wie $+$, $-$, $\ast$, $/$ etc.  Damit sind
Terme wie zum Beispiel $(+~(-~5~3)~17)$ möglich.

Im angewandten $\lambda$-Kalkül kommen zu den Werten aus
Definition~\ref{def:wert} die Basiswerte dazu:
%
\begin{definition}[Werte im angewandten $\lambda$-Kalkül]\label{def:wert-angewandt}
  Im angewandten $\lambda$-Kalkül heißen die Abstraktionen und
  Basiswerte kollektiv \textit{Werte}.  Ein $\lambda$-Term, der kein
  Wert\index{Wert} ist, heißt \textit{Nichtwert\index{Nichtwert}}.
\end{definition}

Damit die primitiven Operationen auch tatsächlich eine Bedeutung
bekommen, muss eine spezielle Reduktionsregel für sie eingeführt
werden:
%
\begin{definition}[$\delta$-Reduktion]\index{delta-Reduktion@$\delta$-Reduktion}
\begin{displaymath}
  (F^k~e_1~\ldots~e_k) \rightarrow_{\delta} F_B^k(e_1, \ldots, e_k)
  \quad e_1, \ldots, e_k \in B
\end{displaymath}
\end{definition}
%
Diese Regel besagt, dass eine primitive Applikation, wenn alle
Operanden Werte sind, durch Anwendung der entsprechenden Operation 
reduziert werden kann.  Damit wird z.B.\ der
obige Beispielterm folgendermaßen reduziert:
%
\begin{displaymath}
  (+~(-~5~3)~17) \rightarrow_{\delta} (+~2~17) \rightarrow_{\delta} 19
\end{displaymath}

\section{Die einfache SECD-Maschine}

Wie schon in Abschnitt~\ref{sec:scheme-auswertung} erwähnt, ist 
der Call-by-Value-$\lambda$-Kalkül\index{Call-by-Value-Reduktion}
ein Modell für die Auswertung der Lehrsprachen und viele andere
Programmiersprachen .
Allerdings ist Definition~\ref{def:call-by-value} strenggenommen etwas
vage: Es wird immer nur der Subterm reduziert, der
"<möglichst weit links innen steht">, aber was das heißt, ist nicht
genau definiert.  Außerdem ist Reduktion zwar ein
mächtiges formales Modell, entspricht aber nicht der
Ausführungsmethode tatsächlicher Programmiersprachen auf echten
Prozessoren.  Ein präzises und echten Maschinen deutlich näheres
Modell ist die \textit{SECD-Maschine\index{SECD-Maschine}}, erfunden
schon in den 60er Jahren von Peter Landin~\cite{Landin1964}, und
seitdem die Grundlage für zahllose Implementierungen von
Call-by-Value-Sprachen.  (Die Darstellung hier ist gegenüber Landins
ursprünglicher Formulierung etwas modernisiert.)

Damit ein Programm aus dem angewandten $\lambda$-Kalkül mit der
SECD-Maschine ausgewertet werden kann, muss es erst einmal in einen
speziellen \textit{Maschinencode\index{Maschinencode}} übersetzt oder
"<compiliert"> werden.  Der Maschinencode besteht, anders als der
$\lambda$-Kalkül, nicht aus geschachtelten Termen, sondern aus einer
Folge von \textit{Instruktionen\index{Instruktion}}.  

\begin{definition}[Maschinencode]\label{def:secd-code}
  In der folgenden Definition ist $I$ die Menge der Instruktionen:
  \begin{grammar}
    \meta{I} \: \meta{B}
    \> \| \meta{V}
    \> \| ap
    \> \| prim$_{F^i}$ \textrm{für alle $F^i \in \Sigma^i$}
    \> \| (\meta{V}, \meta{C})
  \end{grammar}
  %
  Ein Maschinencode-Programm ist eine Folge von Instruktionen:
  %
  \begin{displaymath}
    C = I^\ast
  \end{displaymath}
\end{definition}

Ein Term aus dem angewandten $\lambda$-Kalkül wird mit Hilfe folgender
Funktion in Maschinencode übersetzt:
%
\begin{eqnarray*}
  \llbracket \underline{~} \rrbracket &:& \mathcal{L}_{\lambda{}A}
  \rightarrow C\\
  \llbracket e \rrbracket &\deq&
  \begin{cases}
    b & \textrm{falls $e = b \in B$}\\
    v & \textrm{falls $e = v \in V$}\\
    \llbracket e_0\rrbracket~\llbracket e_1\rrbracket~\mathtt{ap}
    & \textrm{falls $e = (e_0~e_1)$}\\
    \llbracket e_1\rrbracket~\ldots~\llbracket e_k\rrbracket~\mathtt{prim}_{F^k}
    & \textrm{falls $e = (F~e_1~\ldots~e_k)$}\\
    (v, \llbracket e_0\rrbracket) & \textrm{falls $e = \lambda v.e_0$}
  \end{cases}
\end{eqnarray*}
%
Die Übersetzungsfunktion "<linearisiert"> einen $\lambda$-Term.  Zum
Beispiel bedeutet die Übersetzung $\llbracket e_0\rrbracket~\llbracket
e_1\rrbracket~\mathtt{ap}$ für einen Term $(e_0~e_1)$, dass zuerst
$e_0$ ausgewertet wird, danach wird $e_1$ ausgewertet, und schließlich wird die
eigentliche Applikation ausgeführt:  Entsprechend steht $\mathtt{ap}$
für "<Applikation ausführen"> und $\mathtt{prim}_{F^k}$ für "<Primitiv
$F$ ausführen">.  Basiswerte und Variablen werden im Maschinencode
belassen.  Ein $\lambda$-Term wird übersetzt in ein Tupel aus seiner
Variable und dem Maschinencode für seinen Rumpf.

Durch die Linearisierung sind die Instruktionen schon in einer Liste in der
Reihenfolge ihrer Ausführung aufgereiht.  Insbesondere hat die
Linearisierung den Begriff "<links innen"> formalisiert: der jeweils
am weitesten links innen stehende Redex steht in der Liste der
Instruktionen vorn.

Beispiel:
%
\begin{eqnarray*}
  \llbracket \lambda f.\lambda x.\lambda y.f~(+~x~(*~y~2))\rrbracket
  &=&
  (f, \llbracket \lambda x.\lambda y.f~(+~x~(*~y~2))\rrbracket)\\
  &=&
  (f, (x, \llbracket \lambda y.f~(+~x~(*~y 2))\rrbracket))\\
  &=&
  (f, (x, (y, \llbracket f~(+~x~(*~y 2))\rrbracket)))\\
  &=&
  (f, (x, (y, \llbracket f\rrbracket \llbracket (+~x~(*~y
  2))\rrbracket \mathtt{ap})))\\
  &=&
  (f, (x, (y, f \llbracket (+~x~(*~y 2))\rrbracket\mathtt{ap})))\\
  &=&
  (f, (x, (y, f \llbracket x\rrbracket \llbracket (*~y~2)\rrbracket \mathtt{prim}_+~\mathtt{ap})))\\
  &=&
  (f, (x, (y, f~x \llbracket (*~y~2)\rrbracket \mathtt{prim}_+~\mathtt{ap})))\\
  &=&
  (f, (x, (y, f~x \llbracket y\rrbracket \llbracket 2\rrbracket \mathtt{prim}_*~\mathtt{prim}_+~\mathtt{ap})))\\
  &=&
  (f, (x, (y, f~x~y \llbracket 2\rrbracket \mathtt{prim}_*~\mathtt{prim}_+~\mathtt{ap})))\\
  &=&
  (f, (x, (y, f~x~y~2~\mathtt{prim}_*~\mathtt{prim}_+~\mathtt{ap})))\\
\end{eqnarray*}
%
Das Beispiel zeigt deutlich, wie der Rumpf der innersten Abstraktion
in eine lineare Folge von Instruktionen übersetzt wird, die genau der
Call-by-Value-Reduktionsstrategie entspricht: erst $f$ auswerten, dann
$x$, dann $y$, dann das Primitiv $*$ anwenden, dann $+$, und
schließlich die Applikation durchführen.

Nun zur eigentlichen SECD-Maschine~-- sie funktioniert ähnlich wie ein
Reduktionskalkül, operiert aber auf sogenannten
\textit{Maschinenzuständen}: die Maschine überführt also einen
Maschinenzustand durch einen Auswertungsschritt in einen neuen
Maschinenzustand.  Ein Maschinenzustand ist dabei ein 4-Tupel aus der
Menge $S\times E\times C\times D$ (daher der Name der Maschine).  Die
Buchstaben sind deshalb so gewählt, weil $S$ der sogenannte
\textit{Stack\index{Stack}}, $E$ die sogenannte
\textit{Umgebung\index{Umgebung}} beziehungsweise auf englisch das
\textit{Environment\index{Environment}}, $C$ der schon bekannte
Maschinencode beziehungsweise \textit{Code\index{Code}} und $D$ der
sogenannte \textit{Dump\index{Dump}} ist.  Die formalen Definitionen
dieser Mengen sind wie folgt; dabei ist $W$ die Menge der Werte:
%
\begin{eqnarray*}
  S &=& W^\ast\\
  E &=& \mathcal{P}(V\times W)\\
  D &=& (S\times E \times C)^\ast\\
  W &=& B \cup (V\times C\times E)
\end{eqnarray*}
%
Der Stack ist dabei eine Folge von Werten.  In der Maschine sind dies
die Werte der zuletzt ausgewerteten Terme, wobei der zuletzt
ausgewertete Term vorn beziehungsweise "<oben"> steht.  Die Umgebung ist eine
partielle Abbildung von Variablen auf Werte: sie ersetzt die
Substitution in der Reduktionsrelation des $\lambda$-Kalküls.  Anstatt
dass Werte für Variablen eingesetzt werden, merkt sich die Umgebung
einfach, an welche Werte die Variablen gebunden sind.  Erst wenn der
Wert einer Variablen benötigt wird, holt ihn die Maschine aus der
Umgebung.  Der Dump schließlich ist eine Liste früherer Zustände der
Maschine: er entspricht dem Kontext\index{Kontext} im
Substitutionsmodell.

Die Menge $W$ schließlich entspricht dem Wertebegriff aus
Definition~\ref{def:wert-angewandt}: Die Basiswerte gehören dazu,
außerdem Tripel aus $(V\times C\times E)$.  Ein solches Tripel,
genannt \textit{Closure\index{Closure}}~-- repräsentiert den Wert
einer Abstraktion~-- es besteht aus der Variable einer Abstraktion,
dem Maschinencode ihres Rumpfs und der Umgebung, die notwendig ist, um
die Abstraktion anzuwenden: Die Umgebung wird benötigt, damit die
freien Variablen der Abstraktion entsprechend der lexikalischen
Bindung\index{lexikalische Bindung} ausgewertet werden können.  Dies
ist anders als im Substitutionsmodell, wo Variablen bei der
Applikation direkt ersetzt werden und damit verschwinden.  Eine
Closure ist also einfach die Repräsentation einer Funktion.

Im Verlauf der Auswertung werden Umgebungen häufig um neue Bindungen
von einer Variable an einen Wert erweitert.  Dazu ist die Notation
$e[v\mapsto w]$ nützlich.  $e[v\mapsto w]$ konstruiert aus einer
Umgebung $e$ eine neue Umgebung, in der die Variable $v$ an den Wert
$w$ gebunden ist.  Hier ist die Definition:
%
\begin{displaymath}
  e[v\mapsto w] \deq (e \setminus \{ (v, w') | (v, w') \in e \}) \cup \{
    (v, w) \}
\end{displaymath}
%
Es wird also zunächst eine eventuell vorhandene alte Bindung entfernt
und dann eine neue hinzugefügt.

Um einen $\lambda$-Term $e$ in die SECD-Maschine zu "<injizieren">,
wird er in einen Anfangszustand $(\epsilon, \varnothing, \llbracket
e\rrbracket, \epsilon)$ übersetzt.  Dann wird dieser Zustand
wiederholt in die Zustandsübergangsrelation $\hookrightarrow$
gefüttert.  In der folgenden Definition von $\hookrightarrow$ sind
Bezeichner mit einem Unterstrich versehen, wenn es sich um Folgen
handelt, also z.B.\ \underline{s} für einen Stack:
%
\begin{eqnarray}
  \hookrightarrow &\in& \mathcal{P}((S\times E\times C\times D) \times (S\times E\times C\times D))\notag\\
  (\underline{s}, e, b\underline{c}, \underline{d})
  &\hookrightarrow& 
  (b\underline{s}, e, \underline{c}, \underline{d})
  \label{secd:base}
  \\
  (\underline{s}, e, v\underline{c}, \underline{d})
  &\hookrightarrow&
  (e(v)\underline{s}, e, \underline{c}, \underline{d})
  \label{secd:variable}
  \\
  (b_k\ldots b_1 \underline{s}, e, \mathtt{prim}_{F^k}\underline{c}, \underline{d})
  &\hookrightarrow&
  (b\underline{s}, e, \underline{c}, \underline{d})
  \label{secd:prim}
  \\ && \textrm{wobei $F^k\in\Sigma^k$ und $F^k_B(b_1,\ldots,b_k) = b$}\notag
  \\
  (\underline{s}, e, (v, \underline{c'}) \underline{c}, \underline{d})
  &\hookrightarrow&
  ((v, \underline{c'}, e) \underline{s}, e, \underline{c}, \underline{d})
  \label{secd:abstraction}
  \\
  (w (v,\underline{c'}, e') \underline{s}, e, \mathtt{ap}~\underline{c}, \underline{d})
  &\hookrightarrow&
  (\epsilon, e'[v\mapsto w], \underline{c'}, (\underline{s}, e, \underline{c}) \underline{d})
  \label{secd:app}
  \\
  (w, e, \epsilon, (\underline{s'}, e', \underline{c'}) \underline{d})
  &\hookrightarrow&
  (w\underline{s'}, e', \underline{c'}, \underline{d})
  \label{secd:return}
\end{eqnarray}
%
Die Regeln definieren eine Fallunterscheidung nach der ersten
Instruktion der Code"=Komponente des Zustands, beziehungsweise greift die letzte
Regel, wenn der Code leer ist.  Der Reihe nach arbeiten die Regeln wie
folgt:
%
\begin{itemize}
\item Regel~\ref{secd:base} (die
  \textit{Literalregel\index{Literalregel}}) schiebt einen Basiswert
  direkt auf den Stack.
\item Regel~\ref{secd:variable} (die
  \textit{Variablenregel\index{Variablenregel}}) ermittelt den Wert
  einer Variable aus der Umgebung und schiebt diesen auf den Stack.
\item Regel~\ref{secd:prim} ist die
  \textit{Primitivregel\index{Primitivregel}}.  Bei einer primitiven
  Applikation müssen soviele Basiswerte oben auf dem Stack liegen wie
  die Stelligkeit des Primitivs.  Dann ermittelt die Primitivregel das Ergebnis der
  primitiven Applikation und schiebt es oben auf den Stack.
\item Regel~\ref{secd:abstraction} ist die
  \textit{Abstraktionsregel\index{Abstraktionsregel}}: Das Tupel
  $(v,\underline{c'})$ ist bei der Übersetzung aus einer Abstraktion
  entstanden.  Die Regel ergänzt $v$ und $\underline{c'}$ mit
  $e$ zu einer Closure, die auf den Stack geschoben wird.
\item Regel~\ref{secd:app} ist die
  \textit{Applikationsregel\index{Applikationsregel}}: Bei einer
  Applikation müssen oben auf dem Stack ein Wert sowie eine Closure
  liegen.  (Zur Erinnerung: Eine Applikation kann nur ausgewertet
  werden, wenn eine Abstraktion vorliegt.  Abstraktionen werden zu
  Closures ausgewertet.)  In einem solchen Fall "<sichert"> die
  Applikation den aktuellen Zustand auf den Dump, und die Auswertung fährt mit
  einem leeren Stack, der Umgebung aus der Closure~-- erweitert um
  eine Bindung für die Variable~-- und dem Code aus der Closure fort.
\item Regel~\ref{secd:return} ist die
  \textit{Rückkehrregel\index{Rückkehrregel}}: Sie ist anwendbar,
  wenn das Ende des Codes erreicht ist.  Das heißt, dass gerade
  die Auswertung einer Applikation fertig ist.  Auf dem Dump liegt
  aber noch ein gesicherter Zustand, der jetzt "<zurückgeholt"> wird.
\end{itemize}
%
Hier ein Beispiel für den Ablauf der SECD-Maschine für den Term
$(((\lambda x.\lambda y.(+~x~y))~1)~2)$:
% (secd-step*/tex (inject-secd '(((lambda (x) (lambda (y) (+ x y))) 1) 2)))
%
\begin{displaymath}
  \begin{array}{l@{}llll}
&(\epsilon, &\varnothing, &(x, (y, x~y~\mathtt{prim}_+))~1~\mathtt{ap}~2~\mathtt{ap}, &\epsilon)\\
\hookrightarrow{}&((x, (y, x~y~\mathtt{prim}_+), \varnothing), &\varnothing, &1~\mathtt{ap}~2~\mathtt{ap}, &\epsilon)\\
\hookrightarrow{}&(1~(x, (y, x~y~\mathtt{prim}_+), \varnothing), &\varnothing, &\mathtt{ap}~2~\mathtt{ap}, &\epsilon)\\
\hookrightarrow{}&(\epsilon, &\{(x, 1)\}, &(y, x~y~\mathtt{prim}_+), &(\epsilon, \varnothing, 2~\mathtt{ap}))\\
\hookrightarrow{}&((y, x~y~\mathtt{prim}_+, \{(x, 1)\}), &\{(x, 1)\}, &\epsilon, &(\epsilon, \varnothing, 2~\mathtt{ap}))\\
\hookrightarrow{}&((y, x~y~\mathtt{prim}_+, \{(x, 1)\}), &\varnothing, &2~\mathtt{ap}, &\epsilon)\\
\hookrightarrow{}&(2~(y, x~y~\mathtt{prim}_+, \{(x, 1)\}), &\varnothing, &\mathtt{ap}, &\epsilon)\\
\hookrightarrow{}&(\epsilon, &\{(x, 1), (y, 2)\}, &x~y~\mathtt{prim}_+, &(\epsilon, \varnothing, \epsilon))\\
\hookrightarrow{}&(1, &\{(x, 1), (y, 2)\}, &y~\mathtt{prim}_+, &(\epsilon, \varnothing, \epsilon))\\
\hookrightarrow{}&(2~1, &\{(x, 1), (y, 2)\}, &\mathtt{prim}_+, &(\epsilon, \varnothing, \epsilon))\\
\hookrightarrow{}&(3, &\{(x, 1), (y, 2)\}, &\epsilon, &(\epsilon, \varnothing, \epsilon))\\
\hookrightarrow{}&(3, &\varnothing, &\epsilon, &\epsilon)
  \end{array}
\end{displaymath}
%
Die Zustandsübergangsrelation $\hookrightarrow$ ist nun die Grundlage
für die \textit{Auswertungsfunktion\index{Auswertungsfunktion}} der
SECD-Maschine, die für einen $\lambda$-Term dessen Bedeutung
ausrechnet.  Dies ist scheinbar ganz einfach:
%
\begin{eqnarray*}
  \mathit{eval}_\mathit{SECD} & : & \mathcal{L}_{\lambda{}A} \rightarrow B\\
  \mathit{eval}_\mathit{SECD}(e) &= & x \textrm{ wenn } (\epsilon, \varnothing, \llbracket e\rrbracket, \epsilon)
    \hookrightarrow^* (x, e, \epsilon, \epsilon)
\end{eqnarray*}
%
Diese Definition hat jedoch zwei Haken:
%
\begin{itemize}
\item Die Auswertung von $\lambda$-Termen terminiert nicht immer (wie
  zum Beispiel für den "<Endlos"=Term"> $(\lambda x.(x~x))~(\lambda x.(x~x))$), es kommt
  also nicht immer dazu, dass die Zustandsübergangsrelation bei einem
  Zustand der Form $(\epsilon, \varnothing, \llbracket e\rrbracket,
  \epsilon)$ terminiert.
\item Das $x$ aus dieser Definition ist nicht immer ein Basiswert~--
  es kann auch eine Closure sein.
\end{itemize}
%
Der erste Haken sorgt dafür, dass die Auswertungsfunktion nur eine
Relation im Sinne einer "<partiellen Funktion"> ist.  Meist wird
trotzdem von einer Auswertungsfunktion gesprochen.  Beim zweiten
Haken, wenn $x$ eine Closure ist, lässt sich mit dem Resultat nicht
viel anfangen: Um die genaue Bedeutung der Closure herauszubekommen,
müsste sie angewendet werden~-- das Programm ist aber schon fertig
gelaufen.  Es ist also gar nicht sinnvoll, zwischen verschiedenen
Closures zu unterscheiden.  Darum wird für die Zwecke der
Auswertungsfunktion eine Menge $Z$ der \textit{Antworten\index{Antwort}}
definiert, die einen designierten Spezialwert für Closures enthält:
%
\begin{displaymath}
  Z = B \cup \{ \texttt{function} \}
\end{displaymath}
%
Damit lässt sich die Evaluationsfunktion wie folgt definieren:
%
\begin{eqnarray*}
  \mathit{eval}_\mathit{SECD} & \in & \mathcal{L}_{\lambda{}A} \times Z\\
  \mathit{eval}_\mathit{SECD}(e) & = &
  \begin{cases}
    b & \textrm{falls } (\epsilon, \varnothing, \llbracket e\rrbracket, \epsilon)
    \hookrightarrow^* (b, e, \epsilon, \epsilon)\\
    \texttt{function} & \textrm{falls } (\epsilon, \varnothing, \llbracket e\rrbracket, \epsilon)
    \hookrightarrow^* ((v, \underline{c}, e'), e, \epsilon, \epsilon)\\
  \end{cases}
\end{eqnarray*}

\section{\texttt{Quote} und Symbole}
\label{sec:quote}

Dieses Kapitel wird ab hier Gebrauch von einer weiteren
Sprachebene\index{Sprachebene!fortgeschritten} in
\drscheme{} machen, nämlich \texttt{Die Macht der Abstraktion -
  fortgeschritten}.  Diese Ebene muss mit dem \drscheme{}-Menü \texttt{Sprache}
unter \texttt{Sprache auswählen} aktiviert sein, damit die
Programme dieses Kapitels funktionieren.

Die entscheidende Änderung gegenüber den früheren Sprachebenen ist
die Art, mit der die REPL Werte ausdruckt.  (Diese neue Schreibweise,
ermöglicht, die Programme des Interpreters, die als Werte
repräsentiert sind, korrekt auszudrucken.)  Bei Zahlen, Zeichenketten
und booleschen Werten bleibt alles beim alten:
%
\begin{alltt}
5
\evalsto{} 5
"Mike ist doof"
\evalsto{} "Mike ist doof"
#t
\evalsto{} #t
\end{alltt}
%
Bei Listen sieht es allerdings anders aus:
%
\begin{alltt}
(list 1 2 3 4 5 6)
\evalsto{} (1 2 3 4 5 6)
\end{alltt}
%
Die REPL druckt also eine Liste aus, indem sie zuerst eine öffnende
Klammer ausdruckt, dann die Listenelemente (durch Leerzeichen
getrennt) und dann eine schließende Klammer.

Das funktioniert auch für die leere Liste:
%
\begin{alltt}
empty
\evalsto{} ()
\end{alltt}
%
Mit der neuen Sprachebene bekommt außerdem der Apostroph, der dem
Literal\index{Literal} für die leere Liste voransteht, eine erweiterte Bedeutung.
Unter anderem kann der Apostroph benutzt werden, um Literale für
Listen zu formulieren:
%
\begin{alltt}
'(1 2 3 4 5 6)
\evalsto{} (1 2 3 4 5 6)
'(1 #t "Mike" (2 3) "doof" 4 #f 17)
\evalsto{} (1 #t "Mike" (2 3) "doof" 4 #f 17)
'()
\evalsto{} ()
\end{alltt}
%
In der neuen Sprachebene benutzen die Literale und die ausgedruckten
externen Repräsentationen für Listen also die gleiche
Notation\index{Repräsentation}.  Sie unterscheiden sich nur dadurch,
dass beim Literal der Apostroph voransteht.  Der Apostroph funktioniert
auch bei Zahlen, Zeichenketten und booleschen Werten:
%
\begin{alltt}
'5
\evalsto{} 5
'"Mike ist doof"
\evalsto{} "Mike ist doof"
'#t
\evalsto{} #t
\end{alltt}
%
Der Apostroph am Anfang eines Ausdrucks
kennzeichnet diesen also als Literal.  Der Wert des Literals wird 
genauso ausgedruckt, wie es im Programm steht.  (Abgesehen von
Leerzeichen und Zeilenumbrüchen.)  Der Apostroph heißt auf englisch
"<quote">\index{quote@\texttt{quote}}, und deshalb ist diese
Literalschreibweise auch unter diesem Namen bekannt.  Bei Zahlen,
Zeichenketten und booleschen Literalen ist auch ohne Quote klar, dass
es sich um Literale handelt.  Das Quote ist darum bei ihnen rein
optional; sie heißen 
\textit{selbstquotierend}\index{selbstquotierend}.
Bei Listen hingegen sind Mißverständnisse mit anderen
zusammengesetzten Formen möglich, die ja auch mit einer öffnenden Klammer
beginnen: \footnote{Tatsächlich ist die neue Schreibweise für externe
  Repräsentationen die Standard-Repräsentation in Scheme.  Die
  früheren Sprachebenen benutzten die alternative Schreibweise, um die
  Verwirrung zwischen Listenliteralen und zusammengesetzten Formen zu
  vermeiden.}
\begin{alltt}
(1 2 3 4 5 6)
\evalsto{} procedure application: expected procedure, given: 1;
     arguments were: 2 3 4 5 6
\end{alltt}
%
Mit der Einführung von Quote kommt noch eine völlig neue Sorte Werte
hinzu: die \textit{Symbole\index{Symbol}}.  Symbole sind Werte ähnlich wie Zeichenketten und
bestehen aus Text.  Sie unterscheiden sich allerdings dadurch, dass sie
als Literal mit Quote geschrieben und in der REPL ohne
Anführungszeichen ausgedruckt werden:
%
\begin{alltt}
'mike
\evalsto{} mike
'doof
\evalsto{} doof
\end{alltt}
%
Symbole lassen sich mit dem Prädikat
\texttt{symbol?\index{symbol@\texttt{symbol?}}} von anderen Werten
unterscheiden:
%
\begin{alltt}
(symbol? 'mike)
\evalsto{} #t
(symbol? 5)
\evalsto{} #f
(symbol? "Mike")
\evalsto{} #f
\end{alltt}
%
Vergleichen lassen sich Symbole mit \texttt{equal?} (siehe
Abbildung~\ref{scheme:equalp}):

\begin{alltt}
(equal? 'mike 'herb)
\evalsto{} #f
(equal? 'mike 'mike)
\evalsto{} #t
\end{alltt}

Symbole können nicht aus beliebigem Text bestehen.  
Leerzeichen sind zum Beispiel verboten.  Tatsächlich entsprechen die
Namen der zulässigen Symbole genau den Namen von Variablen:
%
\begin{alltt}
'karl-otto
\evalsto{} karl-otto
'mehrwertsteuer
\evalsto{} mehrwertsteuer
'duftmarke
\evalsto{} duftmarke
'lambda
\evalsto{} lambda
'+
\evalsto{} +
'*
\evalsto{} *
\end{alltt}
%
Diese Entsprechung wird in diesem Kapitel noch eine entscheidene Rolle
spielen.  Symbole können natürlich auch in Listen und damit auch in
Listenliteralen vorkommen:
%
\begin{alltt}
'(karl-otto mehrwertsteuer duftmarke)
\evalsto{} (karl-otto mehrwertsteuer duftmarke)
\end{alltt}
%
Mit Hilfe von Symbolen können Werte konstruiert werden, die in der REPL
ausgedruckt wie Ausdrücke aussehen:
%
\begin{alltt}
'(+ 1 2)
\evalsto{} (+ 1 2)
'(lambda (n) (+ n 1))
\evalsto{} (lambda (n) (+ n 1))
\end{alltt}
%
Auch wenn diese Werte wie Ausdrücke so aussehen, sind sie doch ganz
normale Listen: der Wert von \verb|'(+ 1 2)| ist eine Liste mit drei
Elementen: das Symbol \verb|+|, die Zahl \texttt{1} und die Zahl
\texttt{2}.  Der Wert von \verb|'(lambda (n) (+ n 1))| ist ebenfalls
eine Liste mit drei Elementen: das Symbol \verb|lambda|, eine Liste
mit einem einzelnen Element, nämlich dem Symbol \texttt{n}, und einer
weiteren Liste mit drei Elementen: dem Symbol \verb|+|, dem Symbol
\texttt{n} und der Zahl \texttt{1}.

Quote hat noch eine weitere verwirrende Eigenheit:
%
\begin{alltt}
''()
\evalsto{} '()
\end{alltt}
%
Dieses Literal bezeichnet nicht die leere Liste (dann würde nur
\texttt{()} ausgedruckt, ohne Quote), sondern etwas anderes:
%
\begin{alltt}
(cons? ''())
\evalsto{} #t
(first ''())
\evalsto{} quote
(rest ''())
\evalsto{} (())
\end{alltt}
%
Der Wert des Ausdrucks \verb|''()| ist also eine Liste mit zwei
Elementen: das erste Element ist das Symbol \texttt{quote} und das
zweite Element ist die leere Liste.  \texttt{'$t$}
ist selbst also nur syntaktischer Zucker, und zwar für \texttt{(quote
  $t$)}:
%
\begin{alltt}
(equal? (quote ()) '())
\evalsto{} #t
(equal? (quote (quote ())) ''())
\evalsto{} #t
\end{alltt}
%
Quote erlaubt die Konstruktion von Literalen für viele Werte, aber
nicht für alle.  Ein Wert, für den Quote ein Literal konstruieren kann,
heißt \textit{repräsentierbarer
  Wert\index{repräsentierbarer Wert}}.  Die folgende induktive
Definition spezifiziert, was ein repräsentierbarer Wert ist:
%
\begin{itemize}
\item Zahlen, boolesche Werte, Zeichenketten und Symbole sind
  repräsentierbare Werte.
\item Eine Liste aus repräsentierbaren Werten ist ihrerseits ein
  repräsentierbarer Wert. 
\item Nichts sonst ist ein repräsentierbarer Wert.
\end{itemize}

\section{Implementierung der SECD-Maschine}

Die SECD-Maschine ist ein Modell für die Implementierung des
$\lambda$-Kalküls.  Eine solche Implementierung lässt sich in
einfach bauen~-- dieser Abschnitt zeigt, wie.  Der grobe
Fahrplan ergibt sich dabei aus der Struktur der SECD-Maschine selbst:
Nach den obligatorischen Datendefinitionen müssen zunächst Terme in
Maschinencode übersetzt werden.  Dann kommt die
Zustandsübergangsfunktion und schließlich die Auswertungsfunktion an
die Reihe.

\subsection{Datenanalyse}
\label{sec:secd-datenanalyse}

Die erste Aufgabe ist dabei zunächst, wie immer, die Datenanalyse: Am
Anfang stehen die Terme des angewandten $\lambda$-Kalküls.  Eine
geeignete Repräsentation mit Listen und Symbolen lässt dabei die Terme
in der "<fortgeschrittenen"> Sprachebene genau wie entsprechenden
Programm-Terme aussehen:

\noindent\begin{tabular}{lll}
  \texttt{(+ 1 2)} & steht für & $(+~1~2)$\\
  \texttt{(lambda (x) x)} & steht für & $\lambda x.x$\\
  \texttt{((lambda (x) (x x)) (lambda (x) (x x)))} & steht für &
  $(\lambda x.(x~x))~(\lambda x.(x~x))$\\
  etc.
\end{tabular}

Die Datendefinition dafür orientiert sich direkt an
Definition~\ref{def:lambda-angewandt}:
%
\begin{verbatim}
 Ein Lambda-Term ist eins der folgenden:
; - ein Symbol (für eine Variable)
; - eine zweielementige Liste (für eine reguläre Applikation)
; - eine Liste der Form (lambda (x) e) (für eine Abstraktion)
; - ein Basiswert
; - eine Liste mit einem Primitiv als erstem Element
;      (für eine primitive Applikation)
\end{verbatim}
%
Hier die dazu passende Signatur-Definition:
%
\begin{verbatim}
(define term
  (signature
    (mixed symbol
           application
           abstraction
           base
           primitive-application)))
\end{verbatim}
%
Die Signaturen für \texttt{application} etc.\ müssen noch definiert
werden.

Um Verzweigungen über die Sorte \texttt{term} zu ermöglichen, müssen
Prädikate für die einzelnen Teilsorten geschrieben werden.  Diese
können dann für die Definition der entsprechenden Signaturen benutzt
werden.
%
\begin{verbatim}
(: application? (%a -> boolean))
(define application?
  (lambda (t)
    (and (cons? t)
         (not (equal? 'lambda (first t)))
         (not (primitive? (first t))))))

(define application (signature (predicate application?)))

; Prädikat für Abstraktionen
(: abstraction? (%a -> boolean))
(define abstraction?
  (lambda (t)
    (and (cons? t)
         (equal? 'lambda (first t)))))

(define abstraction (signature (predicate abstraction?)))

; Prädikat für primitive Applikationen
(: primitive-application? (%a -> boolean))
(define primitive-application?
  (lambda (t)
    (and (cons? t)
         (primitive? (first t)))))

(define primitive-application (signature (predicate primitive-application?)))
\end{verbatim}
%
Die Definition lässt noch offen, was genau ein "<Basiswert"> und was ein
"<Primitiv"> ist.  Auch hierfür werden noch Datendefinitionen
benötigt, zuerst für Basiswerte.  Der Einfachheit halber beschränkt
sich die Implementierung erst einmal auf boolesche Werte und Zahlen:
%
\begin{verbatim}
; Ein Basiswert ist ein boolescher Wert oder eine Zahl
\end{verbatim}
%
Damit Basiswerte in Fallunterscheidungen von den anderen Arten von
Termen unterschieden werden können, wird ein Prädikat benötigt:
%
\begin{verbatim}
; Prädikat für Basiswerte
(: base? (%a -> boolean))
(define base?
  (lambda (v)
    (or (boolean? v) (number? v))))

(define base (signature (predicate base?)))
\end{verbatim}
%
Als Nächstes sind Primitive gefragt: Am obigen Beispiel ist zu
erkennen, dass z.B.\ \texttt{+} ein Primitiv sein sollte.  Die
Datendefinition für eine kleine beispielhafte Menge von Primitiven ist
wie folgt:
%
\begin{verbatim}
; Ein Primitiv ist eins der Symbole +, -, *, /, =
\end{verbatim}
%
Da die Primitive genau wie die Variablen Symbole sind, stehen die
Primitive als Variablen nicht mehr zur Verfügung:  Alle Symbole, die
keine Primitive sind, sind also Variablen.  Das dazugehörige Prädikat
ist das folgende:
%
\begin{verbatim}
; Prädikat für Primitive
(: primitive? (%a -> boolean))
(define primitive?
  (lambda (s)
    (or (equal? '+ s)
        (equal? '- s)
        (equal? '* s)
        (equal? '/ s)
        (equal? '= s))))

(define primitive (signature (predicate primitive?)))
\end{verbatim}
%
Bevor nun ein die SECD-Maschine einen Term verarbeiten kann, muss
dieser erst in Maschinencode übersetzt werden.  Dabei entsteht aus
Definition~\ref{def:secd-code} direkt Daten- und Signatur-Definitionen
für Instruktionen und Maschinencode:
%
\begin{verbatim}
; Eine Instruktion ist eins der folgenden:
; - ein Basiswert
; - eine Variable
; - eine Applikations-Instruktion
; - eine Instruktion für eine primitive Applikation
; - eine Abstraktion
(define instruction
  (signature
    (mixed base
           symbol
           ap
           tailap
           prim
           abs))

; Eine Maschinencode-Programm ist eine Liste von Instruktionen.
(define machine-code (signature (list-of instruction)))
\end{verbatim}
%
Bei der Definition von Instruktionen ist wieder einiges Wunschdenken
im Spiel.  Basiswerte und Variablen sind wie bei den Termen.  Die
restlichen Fälle werden durch eigene Datendefinitionen abgebildet.
Wie schon bei den leeren Bäumen sind Record-Definitionen ohne Felder
im Spiel, die Fallunterscheidungen möglich machen:
%
\begin{verbatim}
; Eine Applikations-Instruktion ist ein Wert
;   (make-ap)
(define-record-procedures ap
  make-ap ap?
  ())
(: make-ap (-> ap))

; Die Instruktion für eine primitive Applikation
; ist ein Wert
;   (real-make-prim op arity)
; wobei op ein Symbol und arity die Stelligkeit
; ist
(define-record-procedures prim
  real-make-prim prim?
  (prim-operator prim-arity))
(: make-prim (symbol natural -> prim))

; Eine Abstraktions-Instruktion ist ein Wert
;  (make-abs v c)
; wobei v ein Symbol (für eine Variable) und c
; Maschinencode ist
(define-record-procedures abs
  make-abs abs?
  (abs-variable abs-code))
(: make-abs (symbol machine-code -> abs))
\end{verbatim}
%
Da die Stelligkeit eines Primitivs dem Primitiv fest zugeordnet
ist, ist eine Hilfsfunktion nützlich, die bei der Erzeugung eines
Werts der Sorte \texttt{prim} die Stelligkeit ergänzt.
Glücklicherweise haben alle oben eingeführten Primitive die gleiche
Stelligkeit:
%
\begin{verbatim}
; Primitiv erzeugen
(: make-prim (symbol -> prim))
(define make-prim
  (lambda (s)
    (real-make-prim s 2)))
\end{verbatim}
%
Die Einführung von Primitive mit anderen Stelligkeiten ist Gegenstand
von Aufgabe~\ref{aufgabe:prim-arity}.

\subsection{Übersetzung in Maschinencode}

Nun, da sowohl Terme als auch der Maschinencode Datendefinitionen
haben, ist es möglich, die Übersetzung zu programmieren.  Hier sind
Kurzbeschreibung, Signatur und Gerüst:
%
\begin{verbatim}
; Term in Maschinencode übersetzen
(: term->machine-code (term -> machine-code))
(define term->machine-code
  (lambda (e)
    ...))
\end{verbatim}
%
Da es sich bei \texttt{term} um gemischte Daten handelt, muss~-- wie
immer~-- eine Verzweigung den Rumpf der Funktion bilden:
%
\begin{verbatim}
(define term->machine-code
  (lambda (e)
    (cond
      ((symbol? e) ...)
      ((application? e) ...)
      ((abstraction? e) ...)
      ((base? e) ...)
      ((primitive-application? e) ...))))
\end{verbatim}
%
Die Implementierung entspricht in den einzelnen Fällen genau der
Übersetzungsfunktion $\llbracket\underline{~}\rrbracket$. Die Fälle
für Variablen und Basiswerte sind, genau wie dort, trivial:
%
\begin{verbatim}
(define term->machine-code
  (lambda (e)
    (cond
      ((symbol? e) (list e))
      ((base? e) (list e))
      ...)))
\end{verbatim}
%
Bei regulären Applikationen werden
Operator und Operand übersetzt, und das ganze zusammen mit einer
\texttt{ap}-Instruktion zu einer Liste zusammengesetzt:
%
\begin{verbatim}
(define term->machine-code
  (lambda (e)
    (cond
      ...
      ((application? e)
       (append (term->machine-code (first e))
               (append (term->machine-code (first (rest e)))
                       (list (make-ap)))))
      ...)))
\end{verbatim}
%
Bei den primitiven Applikationen werden erst einmal die Operanden in
Maschinencode übersetzt, die Resultate aneinandergehängt, und
schließlich kommt noch eine \texttt{prim}-Instruktion ans Ende:
%
\begin{verbatim}
(define term->machine-code
  (lambda (e)
    (cond
      ...
      ((primitive-application? e)
       (append
        (append-lists
         (map term->machine-code (rest e)))
        (list (make-prim (first e)))))
      ...)))
\end{verbatim}
%
Dieses Stück Code benutzt die Hilfsfunktion \texttt{append-lists}, die
aus einer Liste von Listen eine einzelne Liste macht, indem die
Elemente aneinandergehängt werden:
%
\begin{verbatim}
; die Elemente einer Liste von Listen aneinanderhängen
(: append-lists ((list-of (list-of %a)) -> (list-of %a)))
(define append-lists
  (lambda (l)
    (fold '() append l)))
\end{verbatim}
%
Zurück zur Übersetzung: Eine Abstraktionen wird direkt in eine
\texttt{abs}-Instruktion übersetzt, wobei der Rumpf selbst
noch in Maschinencode übersetzt wird:
%
\begin{verbatim}
(define term->machine-code
  (lambda (e)
    (cond
      ...
      ((abstraction? e)
       (list
        (make-abs (first (first (rest e)))
                  (term->machine-code
                   (first (rest (rest e))))))))))
\end{verbatim}
%

\subsection{Zustandsübergang und Auswertung}
\label{sec:secd-transition}

Da nun alle $\lambda$-Terme in Maschinencode-Programme übersetzt
werden können, ist jetzt die eigentliche SECD-Maschine an der Reihe.
Hier sind erst einmal einige neue Datendefinitionen fällig.  Zunächst
einmal die Menge $S$ der Stacks:
%
\begin{verbatim}
; Ein Stack ist eine Liste von Werten
(define stack (signature (list-of value)))
\end{verbatim}
%
Die Definition von Werten $W$ kommt etwas später an die Reihe.

Umgebungen aus der Menge $E$ sind mathematisch gesehen Mengen aus
Tupeln.  In der Implementierung werden sie dargestellt aus Listen von
\textit{Bindungen\index{Bindung}}, wobei jede Bindung einem Tupel aus
der mathematischen Definition entspricht:

\begin{verbatim}
; Eine Umgebung ist eine Liste von Bindungen.
; Dabei gibt es für jede Variable nur eine Bindung.
(define environment (signature (list-of binding)))

; Eine Bindung (Name: binding) ist ein Wert
;  (make-binding v x)
; wobei v der Name einer Variablen und x der dazugehörige Wert ist.

(define-record-procedures binding
  make-binding binding?
  (binding-variable binding-value))
(: make-binding (symbol value -> binding))
\end{verbatim}
% 
Die leere Umgebung wird öfter benötigt und wird darum schon
vordefiniert:
%
\begin{verbatim}
; die leere Umgebung
(define the-empty-environment empty)
\end{verbatim}
%
Zwei Operationen gibt es für eine Umgebung $e$: die Erweiterung um
eine Bindung $e[v\mapsto w]$ und das Nachschauen einer Bindung
$e(v)$.  Zunächst die Erweiterung: die Implementierung entspricht
genau der mathematischen Definition: zunächst wird eine eventuell
vorhandene Bindung für $v$ entfernt, dann eine neue Bindung
hinzugefügt:
% 
\begin{verbatim}
; eine Umgebung um eine Bindung erweitern
(: extend-environment (environment symbol value -> environment))
(define extend-environment
  (lambda (e v w)
    (cons (make-binding v w)
               (remove-environment-binding e v))))
\end{verbatim}
%
Für das Entfernen der alten Bindung ist die Hilfsfunktion
\texttt{remove-environment-binding} zuständig.  Sie folgt einmal mehr
strikt der Konstruktionsanleitung für Funktionen, die Listen akzeptieren:
% FIXME: Konstruktionsanleitung
\begin{verbatim}
; die Bindung für eine Variable aus einer Umgebung entfernen
(: remove-environment-binding (environment symbol -> environment))
(define remove-environment-binding
  (lambda (e v)
    (cond
      ((empty? e) empty)
      ((cons? e)
       (if (equal? v (binding-variable (first e)))
           (rest e)
           (cons (first e)
                      (remove-environment-binding (rest e) v)))))))
\end{verbatim} 
%
Auch die zweite Operation, das Nachschauen einer Bindung in der
Umgebung, folgt der Konstruktionsanleitung:
%
% FIXME: Konstruktionsanleitung
%
\begin{verbatim}
; die Bindung für eine Variable in einer Umgebung finden
(: lookup-environment (environment symbol -> value))
(define lookup-environment
  (lambda (e v)
    (cond
      ((empty? e) (violation "unbound variable"))
      ((cons? e)
       (if (equal? v (binding-variable (first e)))
           (binding-value (first e))
           (lookup-environment (rest e) v))))))
\end{verbatim}
%
Damit sind die Operationen auf Umgebungen abgeschlossen.  Als Nächstes
sind Dumps an der Reihe: $D$ ist als Folge von Tupeln $S\times E\times
C$ definiert, auch genannt \textit{Frames\index{Frame}}.  Hier sind
Daten- und Record-Definition:
%
\begin{verbatim}
; Ein Dump ist eine Liste von Frames

; Ein Frame ist ein Wert
;  (make-frame s e c)
; wobei s ein Stack, e eine Umgebung und c Maschinencode ist.
(define-record-procedures frame
  make-frame frame?
  (frame-stack frame-environment frame-code))
(: make-frame (stack environment machine-code -> frame))
\end{verbatim}
%
Schließlich fehlt noch eine Repräsentation für die Menge $W$ der
Werte:  Ein Wert ist entweder ein Basiswert oder eine Closure.
Basiswerte wurden bereits in Abschnitt~\ref{sec:secd-datenanalyse}
definiert; es fehlen noch Closures, die Tupel aus $V\times C\times E$
sind.  Hier sind die entsprechenden Definitionen:
%
\begin{verbatim}
; Ein SECD-Wert ist ein Basiswert oder eine Closure
(define value (signature (mixed base closure)))

; Eine Closure ist ein Wert
;  (make-closure v c e)
; wobei v die Variable der Lambda-Abstraktion,
; c der Code der Lambda-Abstraktion
; und e ein Environment ist.
(define-record-procedures closure
  make-closure closure?
  (closure-variable closure-code closure-environment))
(: make-closure (symbol machine-code environment -> closure))
\end{verbatim}
%
Mit Hilfe dieser Definitionen ist es möglich, eine Daten- und eine
Record-Definition für die Zustände der SECD-Maschine anzugeben, also
die Tupel aus $S\times E\times C\times D$:
%
\begin{verbatim}
; Ein SECD-Zustand ist ein Wert
;  (make-secd s e c d)
; wobei s ein Stack, e eine Umgebung, c Maschinencode
; und d ein Dump ist
(define-record-procedures secd
  make-secd secd?
  (secd-stack secd-environment secd-code secd-dump))
(: make-secd (stack environment machine-code dump -> secd))
\end{verbatim}
%
Damit kann es an die Zustandsübergangsfunktion gehen.   Sie wird als
Funktion realisiert, die einen SECD-Zustand akzeptiert und einen neuen
liefert.  Hier sind Kurzbeschreibung, Signatur und Gerüst:
%
\begin{verbatim}
; Zustandsübergang berechnen
(: secd-step (secd -> secd))
(define secd-step
  (lambda (state)
    ...))
\end{verbatim}
%
Entsprechend den Regeln der SECD-Maschine muss der Rumpf der Funktion
eine Verzeigung zwischen den verschiedenen Fällen bei der
Code-Komponente von \texttt{state} sein.  Diese folgen den
Konstruktionsanleitungen für Listen und für gemischte Daten.  Es ist
bereits an den Regeln abzulesen, dass alle Regeln Zugriff auf die
Komponenten von \texttt{state} benötigen.  Für diese werden gleich am
Anfang lokale Variablen angelegt:
%
\begin{verbatim}
(define secd-step
  (lambda (state)
    (let ((stack (secd-stack state))
          (environment (secd-environment state))
          (code (secd-code state))
          (dump (secd-dump state)))
      (cond
        ((cons? code)
         (cond
           ((base? (first code)) ...)
           ((symbol? (first code)) ...)
           ((prim? (first code)) ...)
           ((abs? (first code)) ...)
           ((ap? (first code)) ...)))
        ((empty? code) ...)))))
\end{verbatim}
%
In diesem Gerüst werden nun die Regeln direkt abgebildet.  Hier zur
Erinnerung noch einmal die erste Regel für Basiswerte:
%
\begin{displaymath}
  (\underline{s}, e, b\underline{c}, \underline{d})
  \hookrightarrow
  (b\underline{s}, e, \underline{c}, \underline{d})
\end{displaymath}
%
Hier der passende Code dafür:
%
\begin{verbatim}
(define secd-step
  (lambda (state)
      ...
        (cond
          ((base? (first code))
           (make-secd (cons (first code) stack)
                      environment
                      (rest code)
                      dump))
           ...)
      ...))
\end{verbatim}
%
Hier die Regel für Variablen:
\begin{displaymath}
  (\underline{s}, e, v\underline{c}, \underline{d})
  \hookrightarrow
  (e(v)\underline{s}, e, \underline{c}, \underline{d})
\end{displaymath}
%
Hier der entsprechende Code:
%
\begin{verbatim}
(define secd-step
  (lambda (state)
      ...
        (cond
          ((symbol? (first code))
           (make-secd (cons
                        (lookup-environment environment (first code))
                        stack)
                      environment
                      (rest code)
                      dump))
          ...)
      ...))
\end{verbatim}
%
Die Regel für primitive Applikationen ist etwas aufwendiger:
%
\begin{eqnarray*}
  (b_k\ldots b_1 \underline{s}, e, \mathtt{prim}_{F^k}\underline{c}, \underline{d})
  &\hookrightarrow&
  (b\underline{s}, e, \underline{c}, \underline{d})
  \\ && \textrm{wobei $F^k\in\Sigma^k$ und $F_B(b_1,\ldots,b_k) = b$}
\end{eqnarray*}
%
Für die Implementierung werden Hilfsfunktionen gebraucht, welche die
Argumente vom Stack holen und in der Reihenfolge umdrehen, die
Argumente vom Stack entfernen und schließlich die eigentliche
$\delta$-Transition berechnen:
%
\begin{verbatim}
(define secd-step
  (lambda (state)
      ...
        (cond
          ...
          ((prim? (first code))
           (make-secd (cons
                       (apply-primitive
                         (prim-operator (first code))
                         (take-reverse (prim-arity (first code)) stack))
                       (drop (prim-arity (first code)) stack))
                      environment
                      (rest code)
                      dump))
           ...)
       ...))
\end{verbatim}
%
Die Funktion \texttt{drop} ist gerade die in
Aufgabe~\ref{ex:kartenspiel} geforderte Funktion:
%
\begin{verbatim}
; die ersten Elemente einer Liste weglassen
(: drop (natural (list-of %a) -> (list-of %a)))
\end{verbatim}
%
Die \texttt{take-reverse}-Funktion ist das Pendant zu \texttt{drop},
das die ersten $n$ Elemente einer Liste in umgekehrter Reihenfolge
liefert.  Dies ist am einfachsten über eine endrekursive Hilfsfunktion
zu erledigen~-- aus Kapitel~\ref{cha:accu} ist ja bekannt, dass bei
endrekursiver Konstruktion von Listen gerade immer die Reihenfolge
umgedreht wird:
%
\begin{verbatim}
; die ersten Elemente einer Liste in umgekehrter Reihenfolge berechnen
(: take-reverse (natural (list-of %a) -> (list-of %a)))
(define take-reverse
  (lambda (n l)
    ;; (: loop (natural (list-of %a) (list-of %a) -> (list-of %a)))
    (letrec ((loop (lambda (n l r)
                     (if (= n 0)
                         r
                         (loop (- n 1) (rest l) (cons (first l) r))))))
      (loop n l '()))))
\end{verbatim}
%
Aus einem Primitiv und einer Liste von Argumenten berechnet
\texttt{apply-primitive} das Resultat der primitiven Applikation.
Dabei handelt es sich bei \texttt{primitive} um eine
Fallunterscheidung, der Rumpf der Funktion ist also eine entsprechende
Verzweigung:
%
\begin{verbatim}
; Delta-Transition berechnen
(: apply-primitive (primitive (list-of value) -> value))
(define apply-primitive
  (lambda (p args)
    (cond
      ((equal? p '+)
       (+ (first args) (first (rest args))))
      ((equal? p '-)
       (- (first args) (first (rest args))))
      ((equal? p '=)
       (= (first args) (first (rest args))))
      ((equal? p '*)
       (* (first args) (first (rest args))))
      ((equal? p '/)
       (/ (first args) (first (rest args)))))))
\end{verbatim}
%
Die Regel für Abstraktionen macht aus einer Abstraktion eine Closure:
\begin{displaymath}
  (\underline{s}, e, (v, \underline{c'}) \underline{c}, \underline{d})
  \hookrightarrow
  ((v, \underline{c'}, e) \underline{s}, e, \underline{c}, \underline{d})
\end{displaymath}
%
Der Code macht dies genauso:
%
\begin{verbatim}
(define secd-step
  (lambda (state)
      ...
        (cond
          ...
          ((abs? (first code))
           (make-secd (cons
                        (make-closure (abs-variable (first code))
                                      (abs-code (first code))
                                      environment)
                        stack)
                      environment
                      (rest code)
                      dump)))
       ...)))
\end{verbatim}
%
Hier die Regel für die Applikation:
%
\begin{displaymath}
  (w (v,\underline{c'}, e') \underline{s}, e, \mathtt{ap}~\underline{c}, \underline{d})
  \hookrightarrow
  (\epsilon, e'[v\mapsto w], \underline{c'}, (\underline{s}, e, \underline{c}) \underline{d})
\end{displaymath}
%
Hier der Code dazu:
%
\begin{verbatim}
(define secd-step
  (lambda (state)
      ...
        (cond
          ...
          ((ap? (first code))
           (let ((closure (first (rest stack))))
             (make-secd empty  
                        (extend-environment
                         (closure-environment closure)
                         (closure-variable closure)
                         (first stack))
                        (closure-code closure)
                        (cons
                         (make-frame (rest (rest stack))
                                     environment (rest code))
                         dump))))
          ...)
      ...))
\end{verbatim}
%
Schließlich bleibt noch der Code für die Rückgabe eines Wertes von
einer Funktion.  Hier ist die Regel:
%
\begin{displaymath}
  (w, e, \epsilon, (\underline{s'}, e', \underline{c'}) \underline{d})
  \hookrightarrow
  (w\underline{s'}, e', \underline{c'}, \underline{d})
\end{displaymath}
%
Hier ist der Code dazu:
%
\begin{verbatim}
(define secd-step
  (lambda (state)
      ...
      (cond
        ...
        ((empty? code)
         (let ((f (first dump)))
           (make-secd
            (cons (first stack)
                       (frame-stack f))
            (frame-environment f)
            (frame-code f)
            (rest dump)))))
       ...))
\end{verbatim}
%
Damit die SECD-Maschine in Betrieb genommen werden kann, muss ein Term
$e$ noch in einen Anfangszustand $(\epsilon, \varnothing, \llbracket
e\rrbracket, \epsilon)$ übersetzt werden.  Das erledigt folgende
Hilfsfunktion:
%
\begin{verbatim}
; Aus Term SECD-Anfangszustand machen
(: inject-secd (term -> secd))
(define inject-secd
  (lambda (e)
    (make-secd empty
               the-empty-environment
               (term->machine-code e)
               empty)))
\end{verbatim}
%
Damit lässt sich die Maschine schon ausprobieren:
%
\begin{alltt}
(secd-step (inject-secd '(+ 1 2)))
\evalsto{}#<record:secd (1) () (2 #<record:prim + 2>) ()>
(secd-step (secd-step (inject-secd '(+ 1 2))))
\evalsto{}#<record:secd (2 1) () (#<record:prim + 2>) ()>
(secd-step (secd-step (secd-step (inject-secd '(+ 1 2)))))
\evalsto{}#<record:secd (3) () () ()>
\end{alltt}
%
Es fehlt noch die Auswertungsfunktion $\mathrm{eval}_\mathrm{SECD}$,
die eine Hilfsfunktion benötigt, um die reflexiv-transitive Hülle des
Zustandsübergangs $\hookrightarrow^*$ benötigt:
%
\begin{verbatim}
; bis zum Ende Zustandsübergänge berechnen
(: secd-step* (secd -> secd))
(define secd-step*
  (lambda (state)
    (if (and (empty? (secd-code state))
             (empty? (secd-dump state)))
        state
        (secd-step* (secd-step state)))))
\end{verbatim}
%
Die Auswertungsfunktion orientiert sich direkt an der mathematischen
Definition:
%
\begin{verbatim}
; Evaluationsfunktion zur SECD-Maschine berechnen
(: eval-secd (term -> (mixed value (enum 'function))))
(define eval-secd
  (lambda (e)
    (let ((val (first
                (secd-stack
                 (secd-step* 
                  (inject-secd e))))))
      (if (base? val)
          val
          'proc))))
\end{verbatim}
%
Damit läuft die SECD-Maschine:
%
\begin{alltt}
(eval-secd '(((lambda (x) (lambda (y) (+ x y))) 1) 2))
\evalsto{}3
\end{alltt}

\section{Die endrekursive SECD-Maschine}

Die SECD-Maschine hat einen Schönheitsfehler: Bei endkursiven
Applikationen sollte sie eigentlich, wie in Lehrsprachen-Programmen, keinerlei
zusätzlichen Platz verbrauchen, da kein Kontext anfällt.  Folgende
Beispielauswertung für den Term $(\lambda x.x~x)~(\lambda x.x~x)$
zeigt aber, dass der Zustand mit fortschreitender Auswertung immer
größer wird:
%
% (secd-step*/tex (inject-secd '((lambda (x) (x x)) (lambda (x) (x x)))))
\begin{displaymath}\tiny
  \begin{array}{l@{}llll}
&(\epsilon, &\varnothing, &(x, x~x~\mathtt{ap})~(x, x~x~\mathtt{ap})~\mathtt{ap}, &\epsilon)\\
\hookrightarrow{}&((x, x~x~\mathtt{ap}, \varnothing), &\varnothing, &(x, x~x~\mathtt{ap})~\mathtt{ap}, &\epsilon)\\
\hookrightarrow{}&((x, x~x~\mathtt{ap}, \varnothing)~(x, x~x~\mathtt{ap}, \varnothing), &\varnothing, &\mathtt{ap}, &\epsilon)\\
\hookrightarrow{}&(\epsilon, &\{(x, (x, x~x~\mathtt{ap}, \varnothing))\}, &x~x~\mathtt{ap}, &(\epsilon, \varnothing, \epsilon))\\
\hookrightarrow{}&((x, x~x~\mathtt{ap}, \varnothing), &\{(x, (x, x~x~\mathtt{ap}, \varnothing))\}, &x~\mathtt{ap}, &(\epsilon, \varnothing, \epsilon))\\
\hookrightarrow{}&((x, x~x~\mathtt{ap}, \varnothing)~(x, x~x~\mathtt{ap}, \varnothing), &\{(x, (x, x~x~\mathtt{ap}, \varnothing))\}, &\mathtt{ap}, &(\epsilon, \varnothing, \epsilon))\\
\hookrightarrow{}&(\epsilon, &\{(x, (x, x~x~\mathtt{ap}, \varnothing))\}, &x~x~\mathtt{ap}, &(\epsilon, \{(x, (x, x~x~\mathtt{ap}, \varnothing))\}, \epsilon)~(\epsilon, \varnothing, \epsilon))\\
\hookrightarrow{}&((x, x~x~\mathtt{ap}, \varnothing), &\{(x, (x, x~x~\mathtt{ap}, \varnothing))\}, &x~\mathtt{ap}, &(\epsilon, \{(x, (x, x~x~\mathtt{ap}, \varnothing))\}, \epsilon)~(\epsilon, \varnothing, \epsilon))\\
\hookrightarrow{}&((x, x~x~\mathtt{ap}, \varnothing)~(x, x~x~\mathtt{ap}, \varnothing), &\{(x, (x, x~x~\mathtt{ap}, \varnothing))\}, &\mathtt{ap}, &(\epsilon, \{(x, (x, x~x~\mathtt{ap}, \varnothing))\}, \epsilon)~(\epsilon, \varnothing, \epsilon))\\
\hookrightarrow{}&(\epsilon, &\{(x, (x, x~x~\mathtt{ap}, \varnothing))\}, &x~x~\mathtt{ap}, &(\epsilon, \{(x, (x, x~x~\mathtt{ap}, \varnothing))\}, \epsilon)~(\epsilon, \{(x, (x, x~x~\mathtt{ap}, \varnothing))\}, \epsilon)~(\epsilon, \varnothing, \epsilon))\\
\hookrightarrow{}&((x, x~x~\mathtt{ap}, \varnothing), &\{(x, (x, x~x~\mathtt{ap}, \varnothing))\}, &x~\mathtt{ap}, &(\epsilon, \{(x, (x, x~x~\mathtt{ap}, \varnothing))\}, \epsilon)~(\epsilon, \{(x, (x, x~x~\mathtt{ap}, \varnothing))\}, \epsilon)~(\epsilon, \varnothing, \epsilon))
\\
& \ldots
  \end{array}
\end{displaymath}
%
Damit ist die SECD-Maschine, so wie ist, als Ausführungsmodell für
die Lehrsprachen ungeeignet.  Dieses Manko lässt sich zum Glück reparieren: Die
SECD-Maschine muss endrekursive und "<normale"> Applikationen
unterschiedlich behandeln.  Dazu wird eine neue Instruktion namens
$\mathtt{tailap}$ eingeführt, die wie $\mathtt{ap}$ eine Applikation
durchführt, aber eine endrekursive Applikation signalisiert:
%
\begin{eqnarray*}
  I &=& \ldots\\
  &&\cup \{ \mathtt{tailap} \}
\end{eqnarray*}
%
Als Nächstes muss die Übersetzungsfunktion von Termen in Maschinencode
geändert werden:  Applikationen, die Kontext um sich herum haben,
sollen mit $\mathtt{ap}$ übersetzt werden, solche ohne Kontext mit
$\mathtt{tailap}$.  Da der Applikation allein der Kontext nicht
anzusehen ist, sondern nur dem Term "<drumherum">, wird die
Übersetzungsfunktion $\llbracket \underline{~} \rrbracket$ in zwei
Teile aufgespalten: für einen Term $e$ wird die Auswertungsfunktion $\llbracket\underline{~}\rrbracket$ immer dann benutzt, wenn um $e$ Kontext steht.  Eine
weitere Funktion $\llbracket \underline{~} \rrbracket'$ wird immer
dann aufgerufen, wenn \emph{kein} Kontext drumherum steht.

Kontext entsteht seinerseits immer durch Funktionsapplikationen.  Bei
der Auswertung eines Terms $(e_0~e_1)$ muss \emph{nach} $e_0$ noch
$e_1$ ausgewertet werden, und nach Auswertung von $e_1$ muss noch die
Applikation durchgeführt werden.  Sowohl $e_0$ als auch $e_1$ stehen
in Kontext.  Ähnlich ist es bei den Argumenten von primitiven
Applikationen.

Auf der anderen Seite schneiden Abstraktionen für ihren Rumpf den
Kontext erst einmal ab: Der Rumpf einer Abstraktion kommt schließlich bei der
Auswertung der Abstraktion noch gar nicht zum Zug.  Ob er Kontext hat
oder nicht, entscheidet sich erst bei der Applikation.
Dementsprechend schalten Applikationen und Abstraktionen zwischen den
beiden Funktionen $\llbracket \underline{~} \rrbracket$ und
$\llbracket \underline{~} \rrbracket'$ hin und her:
%
\begin{eqnarray*}
  \llbracket \underline{~} \rrbracket &:& \mathcal{L}_{\lambda{}A} \rightarrow C\\
  \llbracket e \rrbracket &\deq&
  \begin{cases}
    b & \textrm{falls $e = b \in B$}\\
    v & \textrm{falls $e = v \in V$}\\
    \llbracket e_0\rrbracket~\llbracket e_1\rrbracket~\mathtt{ap}
    & \textrm{falls $e = (e_0~e_1)$}\\
    \llbracket e_1\rrbracket~\ldots~\llbracket e_k\rrbracket~\mathtt{prim}_{F^k}
    & \textrm{falls $e = (F~e_1~\ldots~e_k)$}\\
    (v, \llbracket e_0\rrbracket') & \textrm{falls $e = \lambda v.e_0$}
  \end{cases}\\[1ex]
  \llbracket \underline{~} \rrbracket' &:& \mathcal{L}_{\lambda{}A} \rightarrow C\\
  \llbracket e \rrbracket' &\deq&
  \begin{cases}
    b & \textrm{falls $e = b \in B$}\\
    v & \textrm{falls $e = v \in V$}\\
    \llbracket e_0\rrbracket~\llbracket e_1\rrbracket~\mathtt{tailap}
    & \textrm{falls $e = (e_0~e_1)$}\\
    \llbracket e_1\rrbracket~\ldots~\llbracket e_k\rrbracket~\mathtt{prim}_{F^k}
    & \textrm{falls $e = (F~e_1~\ldots~e_k)$}\\
    (v, \llbracket e_0\rrbracket') & \textrm{falls $e = \lambda v.e_0$}
  \end{cases}
\end{eqnarray*}
%
Die Übersetzungsfunktion hat die eigentliche Arbeit geleistet: Jetzt
muss nur noch eine Zustandsübergangsregel her, die $\mathtt{tailap}$
verarbeitet.  Diese ergibt sich direkt aus den Regeln für
$\mathtt{ap}$ und die Rückgabe eines Wertes:  $\mathtt{tailap}$
funktioniert so, wie $\mathtt{ap}$ direkt gefolgt von der
Rückgaberegel.  Hier sind die beiden Regeln noch einmal zur Erinnerung:
%
\begin{eqnarray*}
   (w (v,\underline{c'}, e') \underline{s}, e, \mathtt{ap}~\underline{c}, \underline{d})
  &\hookrightarrow&
  (\epsilon, e'[v\mapsto w], \underline{c'}, (\underline{s}, e, \underline{c}) \underline{d})
  \\
  (w, e, \epsilon, (\underline{s'}, e', \underline{c'}) \underline{d})
  &\hookrightarrow&
  (w\underline{s'}, e', \underline{c'}, \underline{d})
\end{eqnarray*}
%
Da die erste Regel ein neues Dump-Frame erzeugt und die zweite ein
Dump-Frame "<vernichtet">, entfällt diese Arbeit in der Regel für
$\mathtt{tailap}$:
%
\begin{eqnarray*}
  (w (v,\underline{c'}, e') \underline{s}, e, \mathtt{tailap}~\underline{c}, \underline{d})
  &\hookrightarrow&
  (\underline{s}, e'[v \mapsto w], \underline{c'}, \underline{d})
\end{eqnarray*}
%
Damit läuft das Beispiel zwar immer noch endlos, aber immerhin, ohne
immer mehr Platz zu verbrauchen:
%
\begin{displaymath}\scriptsize
  \begin{array}{l@{}llll}
 \hookrightarrow{}&(\epsilon, &\varnothing, &(x, x~x~\mathtt{tailap})~(x, x~x~\mathtt{tailap})~\mathtt{ap}, &\epsilon)\\
\hookrightarrow{}&((x, x~x~\mathtt{tailap}, \varnothing), &\varnothing, &(x, x~x~\mathtt{tailap})~\mathtt{ap}, &\epsilon)\\
\hookrightarrow{}&((x, x~x~\mathtt{tailap}, \varnothing)~(x, x~x~\mathtt{tailap}, \varnothing), &\varnothing, &\mathtt{ap}, &\epsilon)\\
\hookrightarrow{}&(\epsilon, &\{(x, (x, x~x~\mathtt{tailap}, \varnothing))\}, &x~x~\mathtt{tailap}, &(\epsilon, \varnothing, \epsilon))\\
\hookrightarrow{}&((x, x~x~\mathtt{tailap}, \varnothing), &\{(x, (x, x~x~\mathtt{tailap}, \varnothing))\}, &x~\mathtt{tailap}, &(\epsilon, \varnothing, \epsilon))\\
\hookrightarrow{}&((x, x~x~\mathtt{tailap}, \varnothing)~(x, x~x~\mathtt{tailap}, \varnothing), &\{(x, (x, x~x~\mathtt{tailap}, \varnothing))\}, &\mathtt{tailap}, &(\epsilon, \varnothing, \epsilon))\\
\hookrightarrow{}&(\epsilon, &\{(x, (x, x~x~\mathtt{tailap}, \varnothing))\}, &x~x~\mathtt{tailap}, &(\epsilon, \varnothing, \epsilon))\\
\hookrightarrow{}&((x, x~x~\mathtt{tailap}, \varnothing), &\{(x, (x, x~x~\mathtt{tailap}, \varnothing))\}, &x~\mathtt{tailap}, &(\epsilon, \varnothing, \epsilon))\\
\hookrightarrow{}&((x, x~x~\mathtt{tailap}, \varnothing)~(x, x~x~\mathtt{tailap}, \varnothing), &\{(x, (x, x~x~\mathtt{tailap}, \varnothing))\}, &\mathtt{tailap}, &(\epsilon, \varnothing, \epsilon))\\
\hookrightarrow{}&(\epsilon, &\{(x, (x, x~x~\mathtt{tailap}, \varnothing))\}, &x~x~\mathtt{tailap}, &(\epsilon, \varnothing, \epsilon))\\
\hookrightarrow{}&((x, x~x~\mathtt{tailap}, \varnothing), &\{(x, (x, x~x~\mathtt{tailap}, \varnothing))\}, &x~\mathtt{tailap}, &(\epsilon, \varnothing, \epsilon))\\
\hookrightarrow{}&((x, x~x~\mathtt{tailap}, \varnothing)~(x, x~x~\mathtt{tailap}, \varnothing), &\{(x, (x, x~x~\mathtt{tailap}, \varnothing))\}, &\mathtt{tailap}, &(\epsilon, \varnothing, \epsilon))\\
\hookrightarrow{}&(\epsilon, &\{(x, (x, x~x~\mathtt{tailap}, \varnothing))\}, &x~x~\mathtt{tailap}, &(\epsilon, \varnothing, \epsilon))\\
\hookrightarrow{}&((x, x~x~\mathtt{tailap}, \varnothing), &\{(x, (x, x~x~\mathtt{tailap}, \varnothing))\}, &x~\mathtt{tailap}, &(\epsilon, \varnothing, \epsilon))
 \end{array}
\end{displaymath}
Die Implementierung der endrekursiven SECD-Maschine ist Gegenstand von
Übungsaufgabe~\ref{aufgabe:secd-endekursiv-implementieren}.


\section{Der $\lambda$-Kalkül mit Zustand}

Der bisher vorgestellte $\lambda$-Kalkül liefert keinerlei Erklärung
für das Verhalten von Zuweisungen.  Tatsächlich hat sich schon in
Abschnitt~\ref{sec:assignment-problems} angedeutet, dass Zuweisungen
die Formalisierung deutlich erschweren.  Möglich ist es trotzdem, und
dieser Abschnitt zeigt, wie es geht.

Als erstes muss wieder einmal die Sprache des $\lambda$-Kalküls
erweitert werden, diesmal um \texttt{set!}-Ausdrücke:
%
\begin{definition}[Sprache des angewandten $\lambda$-Kalküls mit Zustand
  $\mathcal{L}_{\lambda{}S}$]\index{angewandter $\lambda$-Kalkül}\index{Zustand}
  Sei $V$ eine abzählbare Menge von Variablen.  Sei $B$ eine Menge von
  \textit{Basiswerten\index{Basiswert}} mit $\mathtt{void} \in B$.
  Sei für eine natürliche Zahl $n$ und $i \in \{1, \ldots, n\}$
  jeweils $\Sigma^i$ eine Menge von \textit{$i$-stelligen
    Primitiva\index{Primitivum}}.  Jedem $F^i\in\Sigma^i$ ist eine
  $i$-stellige Funktion $F_B^i: B\times\ldots\times B \rightarrow
  B$~-- ihre \textit{Operation}~-- zugordnet.
  Seit $A$ eine abzählbare Menge von Adressen mit $V\cap A =
  \varnothing$.\index{Adresse}. 

  Die Sprache des angewandten $\lambda$"=Kalküls mit Zustand, die
  Menge der \textit{angewandten $\lambda$-Terme mit Zustand},
  $\mathcal{L}_{\lambda{}S}$\index{L@$\mathcal{L}_{\lambda{}S}$}, ist
  durch folgende Grammatik definiert:
  \begin{grammar}
    \meta{$\mathcal{L}_{\lambda{}S}$} \: \meta{$V$}
    \> \| (\meta{$\mathcal{L}_{\lambda{}S}$} \meta{$\mathcal{L}_{\lambda{}S}$})
    \> \| ($\lambda$\meta{$V$}.\meta{$\mathcal{L}_{\lambda{}S}$})
    \> \| \meta{$B$}
    \> \| (\meta{$\Sigma^1$}~\meta{$\mathcal{L}_{\lambda{}S}$})
    \> \| (\meta{$\Sigma^2$}~\meta{$\mathcal{L}_{\lambda{}S}$}~\meta{$\mathcal{L}_{\lambda{}S}$})
    \> \ldots
    \> \| (\meta{$\Sigma^n$}~\meta{$\mathcal{L}_{\lambda{}S}$}~\ldots~\meta{$\mathcal{L}_{\lambda{}S}$})  \quad \textrm{($n$-mal)}
    \> \| (set! \meta{V} \meta{$\mathcal{L}_{\lambda{}S}$})
  \end{grammar}
  %
\end{definition}
%
Der \texttt{void}-Wert wird als Rückgabewert von
\texttt{set!}-Ausdrücken dienen.

Um Reduktionsregeln für Zuweisungen zu bilden, ist es notwendig, den
Begriff des \textit{Speichers\index{Speicher}} in den $\lambda$-Kalkül
einzuführen: Im $\lambda$-Kalkül mit Zustand stehen Variablen nicht
mehr für Werte, die für sie eingesetzt werden können, sondern für
\textit{Speicherzellen\index{Speicherzelle}}.  Eine Speicherzelle ist
ein Ort im Speicher, der einen Wert aufnimmt, der auch wieder
verändert werden kann.  Dabei wird jede Speicherzelle durch eine
\textit{Adresse\index{Adresse}} identifiziert.  Eine Adresse ist eine
abstrakte Größe, es kommt also gar nicht darauf an, um was für
eine Art Wert es sich handelt~-- im realen Computer ist eine Adresse
in der Regel einfach eine Zahl.  In diesem Abschnitt steht $A$ für die
Menge der Adressen, die abzählbar sein sollte.

Ein Speicher aus der Menge $M$ ist eine Zuordnung zwischen Adressen aus $A$
und Werten\index{Wert}.  Die Werte sind wie schon im normalen
$\lambda$-Kalkül die Basiswerte und die Abstraktionen~-- hier bekommen
sie, weil sie eine Rolle in den Reduktionsregeln spielen, den Namen
$X$:
%
\begin{eqnarray*}
  M &=& \mathcal{P}(A\times X)\\
  X &=& B \cup \{ \lambda v.e | \lambda v.e \in \mathcal{L}_{\lambda{}S} \}
\end{eqnarray*}

Um Reduktionsregeln für den $\lambda$-Kalkül mit Zustand zu
formulieren, muss $\mathcal{L}_{\lambda{}S}$ noch erweitert werden,
damit die Adressen ins Spiel kommen: Adressen werden Terme und sind
auf der linken Seite von Zuweisungen zulässig:
%
\begin{grammar}
  \meta{$\mathcal{L}_{\lambda{}S}$} \: \ldots
  \> \| \meta{A}
  \> \| (set! \meta{A} \meta{$\mathcal{L}_{\lambda{}S}$})
\end{grammar}
%
Adressen tauchen dabei nur als Zwischenschritte bei der Reduktion auf;
sie sind nicht dafür gedacht, dass sie der Programmierer in ein
Programm schreibt.

Da das bisherige Substitutionsprinzip bei Zuweisungen nicht mehr funktioniert, reicht es
nicht, die Reduktionsregeln für den $\lambda$-Kalkül mit Zustand
einfach nur auf Termen auszudrücken: Ein Term, der ja Adressen
enthalten kann, ergibt nur Sinn, wenn er mit einem Speicher kombiniert
wird.  Die Reduktionsregeln überführen somit immer ein Paar, bestehend aus
einem Term und einem Speicher in ein ebensolches Paar.  Hierbei wird der
Einfachheit halber kein Unterschied mehr zwischen den verschiedenen
Arten der Reduktion gemacht:
%
\begin{eqnarray*}
  b, m &\rightarrow& a, m[a\mapsto b] \textrm{ wobei $a$ frisch}
  \\
  \lambda v.e, m &\rightarrow&
  a, m[a\mapsto \lambda v.e]  \textrm{ wobei $a$ frisch}
  \\
  (a_0~a_1), m &\rightarrow& e[v\mapsto a], m[a\mapsto m(a_1)]
  \textrm{ wobei $m(a_0) = \lambda v.e$ und $a$ frisch} 
  \\
  (\mathtt{set!}~a_0~a_1), m &\rightarrow& \mathtt{void}, m[a_0\mapsto m(a_1)]
  \\
  (F^k a_1 \ldots a_k), m &\rightarrow& a, m[a\mapsto F_B(b_1, \ldots, b_k)] \textrm{
    wobei $b_i = m(a_i) \in B$, $a$ frisch}
\end{eqnarray*}
%
Die Formulierung "<$a$ frisch"> bedeutet dabei, dass $a$ eine Adresse
sein sollte, die in $m$ bisher noch nicht benutzt wurde.  Die Operation
$m[a\mapsto x]$ ist ähnlich wie bei Umgebungen definiert: der alte
Speicherinhalt bei $a$ wird zunächst entfernt, und dann eine neue
Zuordnung für $a$ nach $x$ hinzugefügt:
%
\begin{displaymath}
  m[a\mapsto x] \deq (e \setminus \{ (a, x') | (a, x') \in m) \cup \{
    (a, x) \}
\end{displaymath}
%
Die Regeln sind immer noch über Substitution definiert, allerdings
werden für Variablen jetzt nicht mehr Werte sondern Adressen
eingesetzt.  Sie werden, wie beim normalen Call-by-Value-Kalkül auch,
auf Subterme fortgesetzt, die möglichst weit links innen stehen.

Im folgenden Beispiel stehen fettgedruckte Zahlen ${\bf
0}$, ${\bf 1}$ für Adressen. Die Redexe sind jeweils
unterstrichen:
%
{\small
\begin{displaymath}
  \begin{array}{l}
    ((\lambda x.((\lambda y.x) (\mathtt{set!}~x~(+~x~1))))~12), \varnothing
    \\
    \rightarrow({\bf 0}~\underline{12}), \{ ({\bf 0}, \lambda x.((\lambda
    y.x) (\mathtt{set!}~x~(+~x~1)))) \}
    \\
    \rightarrow\underline{({\bf 0}~{\bf 1})}, \{ ({\bf 0}, \lambda x.((\lambda
    y.x) (\mathtt{set!}~x~(+~x~1)))), ({\bf 1}, 12) \}
    \\
    \rightarrow 
    (\underline{(\lambda y.{\bf 2})} (\mathtt{set!}~{\bf
    2}~(+~{\bf 2}~1))),
    \{ ({\bf 0}, \lambda x.((\lambda
    y.x) (\mathtt{set!}~x~(+~x~1)))), ({\bf 1}, 12), ({\bf
    2}, 12)\}\\
    \rightarrow 
    ({\bf 3}~(\mathtt{set!}~{\bf
    2}~(+~{\bf 2}~\underline{1}))),
    \\
    \quad
    \{ ({\bf 0}, \lambda x.((\lambda
    y.x) (\mathtt{set!}~x~(+~x~1)))), ({\bf 1}, 12), ({\bf
    2}, 12), ({\bf 3}, (\lambda y.{\bf 2}) \}
    \\
    \rightarrow 
    ({\bf 3}~(\mathtt{set!}~{\bf
    2}~\underline{(+~{\bf 2}~{\bf 4})})),
    \\
    \quad
    \{ ({\bf 0}, \lambda x.((\lambda
    y.x) (\mathtt{set!}~x~(+~x~1)))), ({\bf 1}, 12), ({\bf
    2}, 12), ({\bf 3}, (\lambda y.{\bf 2}),
    ({\bf 4}, 1) \}\\
    \rightarrow 
    ({\bf 3} \underline{(\mathtt{set!}~{\bf
      2}~{\bf 5})}),
    \\
    \quad
    \{ ({\bf 0}, \lambda x.((\lambda
    y.x) (\mathtt{set!}~x~(+~x~1)))), ({\bf 1}, 12), ({\bf
    2}, 12), ({\bf 3}, (\lambda y.{\bf 2}),
    ({\bf 4}, 1), ({\bf 5}, 13) \}
    \\
    \rightarrow 
    ({\bf 3}~\mathtt{void}),
    \\
    \quad
    \{ ({\bf 0}, \lambda x.((\lambda
    y.x) (\mathtt{set!}~x~(+~x~1)))), ({\bf 1}, 12), ({\bf
    2}, 13), ({\bf 3}, (\lambda y.{\bf 2}),
    ({\bf 4}, 1), ({\bf 5}, 13) \}
    \\
    \rightarrow 
    \underline{({\bf 3}~{\bf 6})},
    \\
    \quad
    \{ ({\bf 0}, \lambda x.((\lambda
    y.x) (\mathtt{set!}~x~(+~x~1)))), ({\bf 1}, 12), ({\bf
    2}, 13), ({\bf 3}, (\lambda y.{\bf 2}),
    ({\bf 4}, 1), ({\bf 5}, 13), ({\bf 6}, \mathtt{void}) \}
    \\
    \rightarrow 
    {\bf 2},
    \\
    \quad    \{ ({\bf 0}, \lambda x.((\lambda
    y.x) (\mathtt{set!}~x~(+~x~1)))), ({\bf 1}, 12), ({\bf
    2}, 13), ({\bf 3}, (\lambda y.{\bf 2}),
    ({\bf 4}, 1), ({\bf 5}, 13), ({\bf 6}, \mathtt{void}) \}
  \end{array}
\end{displaymath}
}
%
Der Endausdruck steht für die Speicherzelle an Adresse ${\bf
2}$, wo der Wert $13$ steht.  Es ist sichtbar, dass die
Auswertungsmaschinerie durch die Einführung von Zustand deutlich
komplizierter wird.

\section{Die SECDH-Maschine}
\index{sec:SECDH-Maschine}

Die SECD-Maschine ist nicht mächtig genug, um den $\lambda$-Kalkül mit
Zustand zu modellieren: Es fehlt ein Speicher.  Darum muss das
Maschinen-Pendant zum $\lambda$-Kalkül mit Zustand um eine
Speicher-Komponente erweitert werden: Heraus kommt die
\textit{SECDH-Maschine}, um die es in diesem Abschnitt geht.

Der Maschinencode für die SECDH-Maschine ist dabei genau wie bei der
SECD-Maschine, nur dass eine spezielle Zuweisungsoperation hinzukommt:
%
\begin{grammar}
  \meta{I} \: \ldots
  \> \| :=
\end{grammar}
%
Die Übersetzungsfunktion produziert diese neue Instruktion bei
\texttt{set!}-Ausdrücken:
%
\begin{eqnarray*}
  \llbracket e \rrbracket &\deq&
  \begin{cases}
    \ldots\\
    v~\llbracket e'\rrbracket~\mathtt{:=}
    & \textrm{falls $e = (\mathtt{set!}~v~e')$}
  \end{cases}
\end{eqnarray*}  
%
Der Begriff der Adresse aus der Menge $A$ wird direkt aus dem Kalkül
übernommen.  Ähnlich wie im Kalkül landen Zwischenergebnisse nicht
mehr direkt auf dem Stack, sondern stattdessen landen ihre Adressen im
Speicher.  Dementsprechend bilden nun Umgebungen Variablen auf
Adressen ab.  Die neue Komponente $H$ ist gerade der Speicher, auch
genannt \textit{Heap\index{Heap}}, der die Adressen auf Werte
abbildet:
%
\begin{eqnarray*}
  S &=& A^{\ast}\\
  E &=& \mathcal{P}(V\times A)\\
  D &=& (S\times E\times C)^{\ast}\\
  H &=& \mathcal{P}(A\times W)\\
  W &=& B \cup (V\times C\times E)
\end{eqnarray*}
%
Die Regeln für die SECDH-Maschine sind analog zu den Regeln für die
SECD-Maschine.  Zwei Hauptunterschiede gibt es dabei:
%
\begin{itemize}
\item Der Heap aus $H$ gehört nun zum Zustand dazu.  Anders als die
  Umgebung wird er nicht bei der Bildung von Closures "<eingepackt">:
  Stattdessen wird der Heap stets linear von links nach rechts durch
  Regeln durchgefädelt.
\item Zwischenergebnisse nehmen stets den Umweg über den Heap: Immer,
  wenn ein neues Zwischenergebnis entsteht, wird es bei einer neuen
  Adresse im Heap abgelegt. Auf dem Stack landen die
  Adressen der Zwischenergebnisse.
\end{itemize}

\begin{eqnarray*}
  \hookrightarrow &\in& \mathcal{P}((S\times E\times C\times D\times H) \times (S\times E\times C\times D\times H))\\
  (\underline{s}, e, b\underline{c}, \underline{d}, h)
  &\hookrightarrow& 
  (a\underline{s}, e, \underline{c}, \underline{d}, h[a \mapsto b] )
  \\ && \textrm{wobei $a$ frisch}
  \\
  (\underline{s}, e, v\underline{c}, \underline{d}, h)
  &\hookrightarrow&
  (e(v)\underline{s}, e, \underline{c}, \underline{d}, h)
  \\
  (a_k\ldots a_1\underline{s}, e, \mathtt{prim}_{F^k}\underline{c},
  \underline{d}, h)
  &\hookrightarrow&
  (a\underline{s}, e, \underline{c}, \underline{d}, h[a\mapsto b] )
  \\ && \textrm{wobei $a$ frisch, $b_i = h(a_i)$ und $F^k\in\Sigma^k$ und $F^k_B(b_1,\ldots,b_k) = b$}
  \\
  (a_1a_0\underline{s}, e, \mathtt{:=} \underline{c}, \underline{d}, h)
  &\hookrightarrow&
  (a\underline{s}, e, \underline{c}, \underline{d}, h[a_0 \mapsto
  h(a_1)][a\mapsto \mathtt{void}] )
  \\ && \textrm{wobei $a$ frisch}
  \\
  (\underline{s}, e, (v, \underline{c'})\underline{c}, \underline{d},
  h)
  &\hookrightarrow&
  (a\underline{s}, e, \underline{c}, \underline{d}, h[a \mapsto (v,
  \underline{c'}, e)] )
  \\ && \textrm{wobei $a$ frisch}
  \\
  (a_1a_0\underline{s}, e, \mathtt{ap}\underline{c}, \underline{d}, h)
  &\hookrightarrow&
  (\epsilon, e'[v\mapsto a], \underline{c'}, (\underline{s}, e,
  \underline{c}) \underline{d}, h[a\mapsto h(a_1)])
  \\ && \textrm{wobei $a$ frisch und $h(a_0) = (v, \underline{c'}, e')$}
  \\
  (a, e, \epsilon, (\underline{s'}, e', \underline{c'}) \underline{d}, h)
  &\hookrightarrow&
  (a\underline{s'}, e', \underline{c'}, \underline{d}, h)
\end{eqnarray*}

Entsprechend muss die Auswertungsfunktion das Endergebnis im Heap nachschauen:

\begin{eqnarray*}
  \mathit{eval}_\mathit{SECD} & \in & \mathcal{L}_{\lambda{}S} \times Z\\
  \mathit{eval}_\mathit{SECD}(e) & = &
  \begin{cases}
    h(a) & \textrm{falls } (\epsilon, \varnothing, \llbracket
    e\rrbracket, \epsilon, \varnothing)
    \hookrightarrow^* (a, e, \epsilon, \epsilon, h), h(a) \in B\\
    \texttt{proc} & \textrm{falls } (\epsilon, \varnothing, \llbracket e\rrbracket, \epsilon,\varnothing)
    \hookrightarrow^* (a, e, \epsilon, \epsilon, h), h(a) = (v, \underline{c}, e')\\
  \end{cases}
\end{eqnarray*}


\section{Implementierung der SECDH-Maschine}


Für die Implementierung der SECDH-Maschine werden einige der
Funktionen wiederverwendet, die für die SECD-Maschine programmiert
wurden.  Zunächst einmal muss~-- genau wie bei der SECD-Maschine~--
erst einmal die Übersetzung von Termen in Maschinencode realisiert
werden.  Zuweisungsterme haben wie in den Lehrsprachen die folgende Form:
%
\begin{alltt}
(set! \(v\) \(e\))
\end{alltt}
%
Das dazu passende Prädikat ist das folgende:
%
\begin{verbatim}
; Prädikat für Zuweisungen
(: assignment? (%a -> boolean))
(define assignment?
  (lambda (t)
    (and (cons? t)
         (equal? 'set! (first t)))))

(define assignment (signature (predicate assignment?)))
\end{verbatim}
%
Mit Hilfe dieser Definition kann die Signatur-Definition von
\texttt{term} erweitert werden:
%
\begin{verbatim}
(define term
  (signature
    (mixed symbol
           application
           abstraction
           base
           primitive-application
           assignment)))
\end{verbatim}
%
Um zu vermeiden, dass Zuweisungen mit regulären Applikationen
verwechselt werden, muss das Prädikat \texttt{application?} erweitert
werden:
%
\begin{verbatim}
(define application?
  (lambda (t)
    (and (cons? t)
         (not (equal? 'set! (first t)))
         (not (equal? 'lambda (first t)))
         (not (primitive? (first t))))))
\end{verbatim}
%
Als Nächstes wird die zusätzliche $\mathtt{:=}$-Instruktion
repräsentiert.  Hier sind Daten- und Record-Definition:
%
\begin{verbatim}
; Eine Zuweisungs-Instruktion ist ein Wert
;  (make-:=)
(define-record-procedures :=
  make-:= :=?
  ())
(: make-:= (-> :=))
\end{verbatim}
%
Die Signatur-Definition für Maschinen-Instruktionen kann um \texttt{:=}
erweitert werden:
%
\begin{verbatim}
(define instruction
  (signature
    (mixed base
           symbol
           ap
           tailap
           prim
           abs
           :=)))
\end{verbatim}

Bei der Übersetzung in Maschinencode kommt in
\texttt{term->machine-code} ein weiterer Zweig hinzu:
%
\begin{verbatim}
; Term in Maschinencode übersetzen
(: term->machine-code (term -> machine-code))
(define term->machine-code
  (lambda (e)
    (cond
      ...
      ((assignment? e)
       (cons (first (rest e))
                  (append (term->machine-code (first (rest (rest e))))
                          (list (make-:=))))))))
\end{verbatim}
%
Wie bei der SECD-Maschine werden die verschiedenen Mengendefinitionen
erst einmal in Daten- und Record-Definitionen übersetzt.  Das ist für
Stacks, Umgebungen und Speicheradressen ganz einfach:
%
\begin{verbatim}
; Ein Stack ist eine Liste aus Adressen.
(define stackh (signature (list-of address)))

; Eine Umgebung bildet Variablen auf Adressen ab.

; Eine Adresse ist eine ganze Zahl.
(define address (signature natural))
\end{verbatim}
%
Die Änderung in der Definition von Umgebungen bedingt eine Änderung
der Signatur von \texttt{make-binding}:
%
\begin{verbatim}
(: make-binding (symbol address -> binding))
\end{verbatim}

%
Bei der Repräsentation des Heaps ist wichtig, dass eine Operation zur
Beschaffung frischer Adressen eingebaut wird.  Aus diesem Grund
enthält der Heap zusätzlich zu den Zellen auch noch einen Zähler mit
der nächsten frischen Adresse:
%
\begin{verbatim}
; Ein Heap ist ein Wert
;   (make-heap s n)
; wobei n die nächste freie Adresse ist und s eine Liste
; von Zellen.
(define-record-procedures heap
  make-heap heap?
  (heap-cells heap-next))
(: make-heap ((list-of cell) natural -> heap))
\end{verbatim}
%
Der leere Heap wird schon einmal vorfabriziert:
%
\begin{verbatim}
(define the-empty-heap (make-heap empty 0))
\end{verbatim}
%
Jede Zelle ordnet einer Adresse einen Wert zu:
%
\begin{verbatim}
; Eine Zelle ist ein Wert
;   (make-cell a w)
; wobei a eine Adresse und w ein Wert ist
(define-record-procedures cell
  make-cell cell?
  (cell-address cell-value))
(: make-cell (address value -> cell))
\end{verbatim}
%
Die Funktion \texttt{heap-store}, erweitert den Heap um eine Zelle
entsprechend der mathematischen Definition:
%
\begin{verbatim}
; Wert im Speicher ablegen
(: heap-store (heap address value -> heap))
(define heap-store
  (lambda (h a w)
    (make-heap (cons (make-cell a w)
                          (remove-cell a (heap-cells h)))
               ...)))
\end{verbatim}
%
Die Ellipse steht für die nächste frische Adresse: Wenn die bisherige
frische Adresse in \texttt{heap-store} belegt wird, so muss eine neue
frische Adresse gewählt werden:
%
\begin{verbatim}
(define heap-store
  (lambda (h a w)
    (make-heap (cons (make-cell a w)
                          (remove-cell a (heap-cells h)))
               (let ((next (heap-next h)))
                 (if (= a next)
                     (+ next 1)
                     next)))))
\end{verbatim}
%
Es fehlt noch die Hilfsfunktion \texttt{remove-cell}:
%
\begin{verbatim}
; Zelle zu einer Adresse entfernen
(: remove-cell (address (list-of cell) -> (list-of cell)))
(define remove-cell
  (lambda (a c)
    (cond
      ((empty? c) empty)
      ((cons? c)
       (if (= a (cell-address (first c)))
           (rest c)
           (cons (first c)
                      (remove-cell a (rest c))))))))
\end{verbatim}
%
Als Nächstes ist die Operation an der Reihe, die den Wert, der an einer
Adresse im Heap gespeichert ist.  Die Funktion \texttt{heap-lookup} benutzt
eine Hilfsfunktion \texttt{cells-lookup}, um in der Liste von Zellen
nach der richtigen zu suchen:
%
\begin{verbatim}
; den Wert an einer Adresse im Heap nachschauen
(: heap-lookup (heap address -> value))
(define heap-lookup
  (lambda (h a)
    (cells-lookup (heap-cells h) a)))

; den Wert an einer Adresse in einer Liste von Zellen nachschauen
(: cells-lookup ((list-of cell) address -> value))
(define cells-lookup
  (lambda (c a)
    (cond
      ((empty? c) (violation "unassigned address"))
      ((cons? c)
       (if (= a (cell-address (first c)))
           (cell-value (first c))
           (cells-lookup (rest c) a))))))
\end{verbatim}
%
Schließlich fehlt noch eine Repräsentation für den $\mathtt{void}$-Wert:
%
\begin{verbatim}
; Ein void-Wert ist ein Wert
;  (make-void)
(define-record-procedures void
  make-void void?
  ())
(: make-void (-> void))
\end{verbatim}
%
Auch hier wird nur ein $\mathtt{void}$-Wert benötigt, der 
vorfabriziert wird:
%
\begin{verbatim}
(define the-void (make-void))
\end{verbatim}
%
Der Zustand für die SECDH-Maschine wird genau wie bei der
SECD-Maschine repräsentiert, ergänzt um die Komponente für den Heap:
%
\begin{verbatim}
; Ein SECDH-Zustand ist ein Wert
;   (make-secd s e c d h)
; wobei s ein Stack, e eine Umgebung, c Maschinencode,
; d ein Dump und h ein Speicher ist.
(define-record-procedures secdh
  make-secdh secdh?
  (secdh-stack secdh-environment secdh-code secdh-dump secdh-heap))
(: make-secdh (stackh environment machine-code dump heap -> secdh))
\end{verbatim}
%
Die Implementierung der Zustandsübergangsfunktion hat exakt die
gleiche Struktur wie die Implementierung der SECD-Maschine und hält
sich eng an die mathematische Definition der Regeln:
%
\begin{verbatim}
; eine Zustandstransition berechnen
(: secdh-step (secdh -> secdh))
(define secdh-step
  (lambda (state)
    (let ((stack (secdh-stack state))
          (environment (secdh-environment state))
          (code (secdh-code state))
          (dump (secdh-dump state))
          (heap (secdh-heap state)))
      (cond
        ((cons? code)
         (cond
           ((base? (first code))
            (let ((a (heap-next heap)))
              (make-secdh
               (cons a stack)
               environment
               (rest code)
               dump
               (heap-store heap a (first code)))))
           ((symbol? (first code))
            (make-secdh
             (cons (lookup-environment environment (first code))
                        stack)
             environment
             (rest code)
             dump
             heap))
           ((prim? (first code))
            (let ((a (heap-next heap)))
              (make-secdh
               (cons a
                          (drop (prim-arity (first code)) stack))
               environment
               (rest code)
               dump
               (heap-store heap a
                           (apply-primitive
                            (prim-operator (first code))
                            (map (lambda (address)
                                   (heap-lookup heap address))
                                 (take-reverse (prim-arity (first code)) stack)))))))
           ((:=? (first code))
            (let ((a (heap-next heap)))
              (make-secdh
               (cons a (rest (rest stack)))
               environment
               (rest code)
               dump
               (heap-store
                (heap-store heap
                            (first (rest stack)) 
                            (heap-lookup heap (first stack)))
                a the-void))))
           ((abs? (first code))
            (let ((a (heap-next heap)))
              (make-secdh
               (cons a stack)
               environment
               (rest code)
               dump
               (heap-store heap a
                           (make-closure (abs-variable (first code))
                                         (abs-code (first code))
                                         environment)))))
           ((ap? (first code))
            (let ((closure (heap-lookup heap (first (rest stack))))
                  (a (heap-next heap)))
              (make-secdh empty
                          (extend-environment
                           (closure-environment closure)
                           (closure-variable closure)
                           a)
                          (closure-code closure)
                          (cons
                           (make-frame (rest (rest stack)) environment (rest code))
                           dump)
                          (heap-store heap a (heap-lookup heap (first stack))))))
           ((tailap? (first code))
            (let ((closure (heap-lookup heap (first (rest stack))))
                  (a (heap-next heap)))
              (make-secdh (rest (rest stack))
                          (extend-environment
                           (closure-environment closure)
                           (closure-variable closure)
                           a)
                          (closure-code closure)
                          dump
                          (heap-store heap a
                                      (heap-lookup heap (first stack))))))))
        ((empty? code)
         (let ((f (first dump)))
           (make-secdh
            (cons (first stack)
                       (frame-stack f))
            (frame-environment f)
            (frame-code f)
            (rest dump)
            heap)))))))
\end{verbatim}    
%
Es bleibt die Auswertungsfunktion, die ebenfalls genau wie bei der
SECD-Maschine realisiert wird:
%
\begin{verbatim}
; aus Term SECDH-Anfangszustand machen
(: inject-secdh (term -> secdh))
(define inject-secdh
  (lambda (e)
    (make-secdh empty
                the-empty-environment
                (term->machine-code e)
                empty
               the-empty-heap)))

; bis zum Ende Zustandsübergänge berechnen
(: secdh-step* (secdh -> secdh))
(define secdh-step*
  (lambda (state)
    (if (and (empty? (secdh-code state))
             (empty? (secdh-dump state)))
        state
        (secdh-step* (secdh-step state)))))


; Evaluationsfunktion zur SECD-Maschine berechnen
(: eval-secdh (term -> (mixed value (enum 'function))))
(define eval-secdh
  (lambda (e)
    (let ((final (secdh-step* (inject-secdh e))))
      (let ((val (heap-lookup (secdh-heap final)
                              (first (secdh-stack final)))))
        (if (base? val)
            val
            'proc)))))
\end{verbatim}


\section*{Übungsaufgaben}

\begin{aufgabe}
  Übersetzen Sie folgende Lambda-Terme in die Zwischenrepräsentation
  der SECD-Maschine:
  
  \begin{enumerate}
  \item \((\lambda xy.(+\:x\:y))\:(*\:5\:6)\:23\)

  \item \((\lambda x.(!\:x))\:(\lambda xy.(\&\&\:x\:y))\:((\lambda xy.(>\:x\:y))\:23~42)\:true\)

   
  \item \((\lambda xy.\:y\:x\:x)\:(\lambda z.\:z)\:(\lambda yz.\:(y\:y)\:(y\:z))\)

  \end{enumerate}

  Dabei steht $!$ für das boolesche \texttt{not} und $\&\&$ für das
  boolesche \texttt{and}.
\end{aufgabe}

\begin{aufgabe}
 Betrachten Sie folgendes SECD-Programm:
 
 \newcommand{\tuple}[2]{\ensuremath{(#1,#2)}}

 \begin{center}
   \tuple{f}{\tuple{x}{\tuple{y}{f\:x\:\texttt{ap}\:y\:\texttt{ap}}}}\:\tuple{a}{\tuple{b}{a\:b\:\texttt{prim}_+}}\:\texttt{ap}\:23\:\texttt{ap}\:42\:\texttt{ap}
 \end{center}
 
 \begin{enumerate}
   \item Übersetzen Sie das SECD-Programm in den entsprechenden
   $\mathcal{L}_{\lambda{}A}$-Term.
   \item Werten Sie das SECD-Programm aus und geben Sie die einzelnen
     Auswertungsschritte an!
 \end{enumerate}
  
\end{aufgabe}

\begin{aufgabe}\label{aufgabe:secd-endekursiv-implementieren}
  Erweitern Sie die Implementierung der SECD-Maschine um korrekte
  Behandlung der Endrekursion!  Erweitern Sie dazu zunächst die
  Datendefinition für Maschinencode. Implementieren Sie dann die
  Übersetzung von $\lambda$-Termen für die endrekursive SECD-Maschine.
  Erweitern Sie schließlich die Zustandsübergangsfunktion um einen
  Fall für die $\mathtt{tailap}$-Instruktion.
\end{aufgabe}

\begin{aufgabe}
  Die um Endrekursion erweiterte SECD-Maschine führt eine neue
  Maschinencode"=Instruktion \texttt{tailap} ein.  Dies ist aber nicht
  unbedingt nötig.  Formulieren Sie die Zustandsübergangsregeln der
  SECD-Maschine mit Endrekursion so um, dass die Funktionalität, also
  insbesondere die richtige Behandlung endrekursiver Applikationen,
  auch ohne das das zusätzliche Schlüsselwort \texttt{tailap}
  erhalten bleibt.
\end{aufgabe}

\begin{aufgabe}
  Zeigen Sie in der um Endrekursion erweiterten
  SECD-Maschine, dass \texttt{tailap} immer am Ende steht, also
  tatsächlich keinen Kontext besitzt.
\end{aufgabe}

\begin{aufgabe}\label{aufgabe:prim-arity}
  Erweitern Sie die SECD-Maschine um Primitive anderer Stelligkeiten,
  z.B.\ \texttt{abs} oder \texttt{odd?}.
\end{aufgabe}

\begin{aufgabe}
  Ändern Sie die Implementierung der SECDH-Maschine dahingehend, dass
  sie Endrekursion korrekt behandelt.
\end{aufgabe}

\begin{aufgabe}
  Abstrahieren Sie über \texttt{remove-environment-binding} und
  \texttt{remove-cell}.
\end{aufgabe}

\begin{aufgabe}
  Erweitern Sie den angewandten $\lambda$-Kalkül um Abstraktionen und
  Applikationen mit mehr als einem Parameter.  Erweitern Sie die
  SECD-Maschine und ihre Implementierung entsprechend.
\end{aufgabe}


\begin{aufgabe}
  Erweitern Sie den angewandten $\lambda$-Kalkül um binäre
  Verzweigungen analog zu \texttt{if}.  Erweitern Sie entsprechend die
  SECD-Maschine und ihre Implementierung.
\end{aufgabe}

\begin{aufgabe}
  \texttt{Begin} lässt sich im angewandten $\lambda$-Kalkül als
  syntaktischer Zucker auffassen: Wie müssten \texttt{begin}-Ausdrücke
  in die Sprache des Kalküls übersetzt werden?
\end{aufgabe}

\begin{aufgabe}
  Anstatt Umgebungen durch Listen von Bindungen zu repräsentieren, ist
  es auch möglich, Funktionen zu verwenden, so dass
  \texttt{lookup-environment} folgendermaßen aussieht:
\begin{verbatim}
(define lookup-environment
  (lambda (e v)
    (e v)))
\end{verbatim}
  Ergänzen Sie eine passende Definition für
  \texttt{extend-environment}.
\end{aufgabe}

\begin{aufgabe}
  Auf den ersten Blick erscheint es etwas aufwendig, jedesmal bei der
  Auswertung einer Abstraktion die gesamte Umgebung in die Closure
  einzupacken.  Was würde sich ändern, wenn dieser Schritt weggelassen
  würde, Closures also nur Variable und Maschinencode für den Rumpf
  enthalten würden?  Formulieren Sie die entsprechenden Regeln für die
  SECD-Maschine und ändern Sie die Implementierung entsprechend.
  Funktioniert die SECD-Maschine nach der Änderung noch korrekt?
\end{aufgabe}

%%% Local Variables: 
%%% mode: latex
%%% TeX-master: "i1"
%%% End: 

