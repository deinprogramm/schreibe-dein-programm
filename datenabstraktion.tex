% Diese Datei ist Teil des Buchs "Schreibe Dein Programm!"
% Das Buch ist lizensiert unter der Creative-Commons-Lizenz
% "Namensnennung - Weitergabe unter gleichen Bedingungen 4.0 International (CC BY-SA 4.0)"
% https://creativecommons.org/licenses/by-sa/4.0/deed.de

\chapter{Datenabstraktion}
\label{chap:datenabstraktion}

In Kapitel~\ref{cha:trees} haben gesehen, dass Bäume beim Suchen
helfen können, und zwar in einer Vielzahl von Situationen: Ob es nun
um Mitgliedschaft in einer Band geht, das Finden von Telefonnummern
oder Steuerunterlagen.  Dass aber ausgerechnet Bäume dafür
verantwortlich sind, Daten effizient zu finden, ist aber aus Sicht der
konkreten Anwendung nebensächlich.  Es gibt neben Bäumen eine Vielzahl
von Datenstrukturen für effiziente Suche.  Je nach Anwendung könnte es
sogar sinnvoll sein, nachträglich eine Datenstruktur durch eine andere
zu ersetzen, etwa weil sie schneller ist oder weniger Speicherplatz
verbraucht.  So wie wir bisher programmiert haben, ist das aber recht
aufwendig, unter anderem weil Funktionen wie
\lstinline{search-tree-insert} die Datenstruktur im Namen tragen.  In
diesem Kapitel geht es darum, es leicher zu machen, eine Datenstruktur
durch eine andere zu ersetzen~-- mit einer Technik namens
\textit{Datenabstraktion}.\index{Datenabstraktion}

Dieses Kapitel befindet sich noch in der Entstehung.

%%% Local Variables: 
%%% mode: latex
%%% TeX-master: "i1"
%%% End: 

