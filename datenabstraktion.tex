% Diese Datei ist Teil des Buchs "Schreibe Dein Programm!"
% Das Buch ist lizensiert unter der Creative-Commons-Lizenz
% "Namensnennung - Weitergabe unter gleichen Bedingungen 4.0 International (CC BY-SA 4.0)"
% https://creativecommons.org/licenses/by-sa/4.0/deed.de

\chapter{Datenabstraktion}
\label{chap:datenabstraktion}

\index{Datenabstraktion}

In Kapitel~\ref{cha:trees} haben wir gesehen, dass Bäume beim Suchen
helfen können, und zwar in einer Vielzahl von Situationen: Ob es nun
um Mitgliedschaft in einer Band geht, das Finden von Telefonnummern
oder Steuerunterlagen.  Dass aber ausgerechnet Bäume dafür
verantwortlich sind, Daten effizient zu finden, ist aber aus Sicht der
konkreten Anwendung nebensächlich.  Es gibt neben Bäumen eine Vielzahl
von Datenstrukturen für effiziente Suche.  Je nach Anwendung könnte es
sogar sinnvoll sein, nachträglich eine Datenstruktur durch eine andere
zu ersetzen, etwa weil sie schneller ist oder weniger Speicherplatz
verbraucht.  So wie wir bisher programmiert haben, ist das aber recht
aufwendig, unter anderem weil Funktionen wie
\lstinline{search-tree-insert} die Datenstruktur im Namen tragen.  In
diesem Kapitel geht es darum, es leicher zu machen, eine Datenstruktur
durch eine andere zu ersetzen~-- mit einer Technik namens
\textit{Datenabstraktion}.\index{Datenabstraktion}

% \section{So viele Mengen}

% \mentioncode{datenabstraktion/list-set.rkt}

% In Kapitel~\ref{cha:trees} auf Seite~\pageref{cha:trees} haben wir uns
% intensiv mit Bäumen beschäftigt in verschiedenen Ausprägungen.  Dazu
% gehörten Suchbäume, die wir in Abschnitt~\ref{sec:suchbaeume} auf
% Seite~\pageref{ref:sec:suchbaeume} eingeführt haben und in
% Abschnitt~\ref{sec:balancierte-suchbaeume} auf
% Seite~\pageref{sec:balancierte-suchbaeume} dann nochmal balanciert
% haben.  Folgende Funktionen haben wir geschrieben:
% %
% \begin{itemize}
% \item \lstinline{search-tree-member?} und
%   \lstinline{search-tree-insert} für "<normale"> Suchbäume
% \item \lstinline{sized-search-tree-member?} und
%   \lstinline{balanced-search-tree-insert} für balancierte Suchbäume
% \end{itemize}
% %
% Von außen betrachtet machen die beiden Funktionspaare jeweils das
% gleiche: \lstinline{search-tree-member?} und
% \lstinline{sized-search-tree-member?} sowie
% \lstinline{search-tree-insert} und
% \lstinline{balanced-search-tree-insert} haben jeweils entsprechende
% Signaturen, mit dem einzigen Unterschied, dass dort einmal
% \lstinline{search-tree-of} und das andere Mal
% \lstinline{sized-search-tree-of} steht.

% Beide Funktionspaare benutzen Bäume, um \textit{Mengen}\index{Menge}
% zu realisieren, also Datenstrukturen mit Elementen, bei denen nur
% zählt, ob ein Element "<drin"> ist oder eben nicht.  Man spricht auch
% davon, dass die Suchbäume die Idee der Menge
% \textit{implementieren}.\index{Implementation}

% Diesen beiden Implementationen der Mengenidee fügen wir nun noch eine
% deutlich einfachere hinzu, diesmal auf Basis von Listen~--
% "<Listenmengen"> sozusagen.\index{Listenmenge} Wir wollen auch
% diesesmal Funktionen schreiben, deren Signatur der Signatur von
% \lstinline{search-tree-member?} und \lstinline{search-tree-insert}
% entspricht.  Wir können nicht direkt die Funktion \lstinline{member?}
% aus Abschnitt~\ref{func:memberp} auf Seite~\ref{page:memberp}
% benutzen, weil sie als Eingabe die Gleichheitsfunktion auf den
% Elementen der Menge benötigt, und damit eine andere Signatur hat als
% \lstinline{search-tree-member?}.  Wir müssen deshalb, ähnlich wie bei
% der Definition von \lstinline{search-tree-of}, die Gleichheitsfunktion
% mit in die Datendefinition der Mengen aufnehmen.  Die sieht so aus:
% %
% \begin{lstlisting}
% ; Eine Listenmenge besteht aus:
% ; - Gleichheitsfunktion
% ; - einer Liste der Elemente
% (define-record (list-set-of element)
%   make-list-set
%   list-set?
%   (element-=?-function (element element -> boolean))
%   (list-set-list (list-of element)))
% \end{lstlisting}
% %
% Hier eine Beispielmenge:
% %
% \begin{lstlisting}
% ; Menge mit den Elementen 1 2 3 4
% (define list-set1
%   (make-list-set = (list 1 2 3 4)))
% \end{lstlisting}
% %
% Um festzustellen, ob ein Wert Element einer Listenmenge ist, benutzen
% wir die \lstinline{member?}-Funktion aus dem Bäume-Kapitel, die wir
% hier nicht noch einmal abdrucken:
% %
% \begin{lstlisting}
% ; Feststellen, ob Wert Element einer Menge ist
% (: list-set-member? (%a (list-set-of %a) -> boolean))

% (check-expect (list-set-member? 2 list-set1) #t)
% (check-expect (list-set-member? 5 list-set1) #f)

% (define list-set-member?
%   (lambda (value list-set)
%     (member? (element-=?-function list-set)
%              value (list-set-list list-set))))
% \end{lstlisting}
% %
% Bei der Funktion zum Einfügen von Wert unterscheiden wir zwischen
% Mengen, die den Wert schon enthalten und solchen, bei denen er noch
% kein Element ist:
% %
% \begin{lstlisting}
% ; Element in Menge einfügen
% (: list-set-insert (%a (list-set-of %a) -> (list-set-of %a)))

% (check-expect (list-set-member? 5 (list-set-insert 5 list-set1)) #t)
% (check-expect (list-set-member? 5 (list-set-insert 3 list-set1)) #f)

% (define list-set-insert
%   (lambda (value list-set)
%     (if (list-set-member? value list-set)
%         list-set
%         (make-list-set (element-=?-function list-set)
%                        (cons value (list-set-list list-set))))))
% \end{lstlisting}
% %
% Auf diese beiden Funktionen können wir auch Eigenschafts-Tests
% definieren.  Wir können zum Beispiel die Suchbaum-Tests aus
% Abschnitt~\ref{sec:suchbaeume-testen} auf
% Seite~\pageref{sec:suchbaeume-testen} übernehmen und anpassen.  Dazu
% benötigen wir eine Hilfsfunktion analog zu
% \lstinline{list->balanced-search-tree}, die sukzessive alle Elemente
% einer Liste in eine Listenmenge einfügt:
% %
% \begin{lstlisting}
% ; Aus allen Elementen einer Liste eine Listenmenge machen
% (: list->list-set ((%a %a -> boolean)
%                    (list-of %a) -> (list-set-of %a)))

% (define list->list-set
%   (lambda (= elements)
%     (fold (make-list-set = empty)
%           list-set-insert
%           elements)))
% \end{lstlisting}
% %
% Damit sehen die Tests (die auch die Hilfsfunktion \lstinline{every?}
% von Seite~\pageref{func:everyp} benutzen) so aus:
% %
% \begin{lstlisting}
% (check-property
%  (for-all ((elements (list-of natural)))
%    (let ((list-set (list->list-set = elements)))
%      (every? (lambda (element)
%                (list-set-member? element list-set))
%              elements))))

% (check-property
%  (for-all ((elements (list-of natural))
%            (element natural))
%    (==> (not (member? = element elements))
%         (not (list-set-member? element (list->list-set = elements))))))
% \end{lstlisting}
% %
% Jetzt haben wir also drei Implementierungen der Mengen-Idee, die sich
% in den Signaturen der zugehörigen Funktionen ähneln.
% Mantra~\ref{ref:abstraktion} auf Seite~\pageref{mantra:abstraktion}
% rät uns zu abstrahieren.

% Allerdings geht es hier um eine andere Art der Abstraktion als
% sonst~-- bisher haben wir immer über ähnlichen Funktionen
% abstrahiert.  Die Funktionen \lstinline{search-tree-insert},
% \lstinline{balanced-search-tree-insert} und \lstinline{list-set-insert}
% ähneln sich allerdings kaum.  Wir könnten deshalb nicht davon
% profitieren, dass wir wiederholten Code nur einmal schreiben.  Wozu
% also abstrahieren?

% Nehmen wir mal an, wir brauchen in unserem Programm Mengen, um unsere
% Arbeit zu erledigen:  Zum Beispiel, um die Einlasskontrolle bei einem
% Guns"=N'"=Roses-Konzert im Jahr 2001 durchzuführen und darauf zu achten,
% dass Slash und seine Freude nicht hineinkommen: Wir stecken alle
% unerwünschten Personen in eine Menge und überprüfen bei jeder
% Besucherin und jedem Besucher, ob sie in der Menge ist.

% Aber welche der Mengen-Implementierungen sollen wir nehmen?  Sie haben
% unterschiedliche Vor- und Nachteile, was die Laufzeit der beiden
% Funktionen betrifft:
% %
% \begin{itemize}
% \item Bei Suchbäumen ist das Einfügen schnell und die
%   Element"=Überprüfung schnell, solange der Suchbaum nicht entartet.
% \item Bei balancierten Suchbäumen ist das Einfügen etwas langsamer,
%   dafür kann der Suchbaum nicht entarten.  Die Suche ist also
%   tendenziell schneller.
% \item Bei Listenmengen ist die Implementierung besonders einfach und
%   bei sehr kleinen Mengen ist sie deshalb auch schneller als die
%   Bäume.
% \end{itemize}
% %
% Welche von den Implementierungen richtig ist, hängt davon ab, wieviele
% Freunde Slash hat (beziehungsweise damals hatte).  Außerdem kann sich
% diese Zahl auch noch ändern, und dann wäre die Implementierung, die
% vorher noch die "<richtige"> war es vielleicht plötzlich nicht mehr.
% Mit den bisherigen Mitteln müssten wir uns aber entscheiden, zum
% Beispiel für Listenmengen:
% %
% \begin{lstlisting}
% ; Menge mit Slashs Freunden 2001
% (define friends-of-slash-2001
%   (list->list-set
%    string=?
%    "Duff McKagan"
%    "Matt Sorum"
%    "Dave Kushner"
%    "Scott Weiland"))

% ; Ist jemand Freund von Slash im Jahr 2001?
% (: friend-of-slash? (string -> boolean))

% (check-expect (friend-of-slash? "Duff McKagan") #t)
% (check-expect (friend-of-slash? "Buckethead") #f)

% (define friend-of-slash?
%   (lambda (person)
%     (list-set-member? person friends-of-slash)))
% \end{lstlisting}
% %
% Wollen wir nun ändern, welche Implementierung von Mengen die Funktion
% \lstinline{friends-of-slash?}  benutzt, müssten wir sie ändern.  Wenn
% noch mehr Funktionen hinzukommen, die auf
% \lstinline{friends-of-slash-2001} zugreifen, müssten wir jede einzelne
% davon ändern.  Das können wir durch Abstraktion vermeiden.

% \section{Daten und Methoden}

% \mentioncode{datenabstraktion/set.rkt}

% Wir wollen also abstrahieren über Suchbäume, balancierte Suchbäume und
% Listenmengen.  Ähnlich sind jeweils die Signaturen der beiden
% Funktionen:
% %
% \begin{lstlisting}
% (: search-tree-member?
%    (%a (search-tree-of %a)       -> boolean))
% (: sized-search-tree-member?
%    (%a (sized-search-tree-of %a) -> boolean))
% (: list-set-member?
%    (%a (list-set-of %a)          -> boolean))

% (: search-tree-insert
%    (%a (search-tree-of %a)       -> (search-tree-of %a)))
% (: balanced-search-tree-insert
%    (%a (sized-search-tree-of %a) -> (sized-search-tree-of %a)))
% (: list-set-insert
%    (%a (list-set-of %a)          -> (list-set-of %a)))
% \end{lstlisting}
% %
% Wenn wir abstrahieren wollen, müssen wir für
% \lstinline{search-tree-of}, \lstinline{sized-search-tree-of} und
% \lstinline{list-set-of} einen neuen, abstrakten Namen finden.  Wir
% wollen also Funktionen mit folgenden Signaturen schreiben:
% %
% \begin{lstlisting}
% (: set-member? (%a (set-of %a) -> boolean))
% (: set-insert (%a (set-of %a) -> (set-of %a)))
% \end{lstlisting}
% %
% Ein Ansatz für die Vereinigung der drei Signaturen wäre, sie als
% gemischte Daten zu betrachten:
% %
% \begin{lstlisting}
% (define set-of
%   (lambda (element)
%     (mixed (search-tree-of element)
%            (sized-search-tree-of element)
%            (list-of-of element))))
% \end{lstlisting}
% %
% Entsprechend könnten wir dann \lstinline{set-member?} und
% \lstinline{set-insert} folgendermaßen programmieren:
% %
% \begin{lstlisting}
% ; Feststellen, ob Wert Element einer Menge ist
% (: set-member? (%a (set-of %a) -> boolean))

% (define set-member?
%   (lambda (value set)
%     (cond
%       ((search-tree? set) (search-tree-member? value set))
%       ((sized-search-tree? set) (sized-search-tree-member? value set))
%       ((list-set? set) (list-set-member? value set)))))
    

% ; Element in Menge einfügen
% (: set-insert (%a (set-of %a) -> (set-of %a)))

% (define set-insert
%   (lambda (value set)
%     (cond
%       ((search-tree? set) (search-tree-insert value set))
%       ((sized-search-tree? set) (balanced-search-tree-insert value set))
%       ((list-set? set) (list-set-insert value set)))))
% \end{lstlisting}
% %
% Diese Methode erlaubt uns, die Einlassprüfung beim
% Guns-N'-Roses-Konzert unabhängig von der Implementierung von Mengen zu
% realisieren:
% %
% \begin{lstlisting}
% (define friend-of-slash?
%   (lambda (person)
%     (set-member? person friends-of-slash)))
% \end{lstlisting}
% %
% Diese Methode hat einen Nachteil: Wenn eine weitere Implementierung
% der Mengen-Idee hinzukommt, müssen wir alle diese Definitionen ändern.
% Das ist gar nicht so unwahrscheinlich, weil es noch andere
% Datenstrukturen gibt, die schnellen Umgang mit Mengen erlauben.

% Wir wollen darum eine andere Art von Abstraktion ausprobieren, bei der
% wir nicht primär über Funktionen abstrahieren, sondern über Daten: die
% Datenabstraktion.\index{Datenabstraktion}  Wir abstrahieren dafür über
% den Daten-Definitionen der verschiedenen Mengen-Implementierungen
% beziehungsweise den zugehörigen Record-Typen.  Hier sind sie:
% %
% \begin{lstlisting}
% (define-record (search-tree-of element)
%   make-search-tree search-tree?
%   (search-tree-label-=?-function (element element -> boolean))
%   (search-tree-label-<?-function (element element -> boolean))
%   (search-tree-tree (tree-of false element)))

% (define-record (sized-search-tree-of element)
%   make-sized-search-tree sized-search-tree?
%   (sized-search-tree-label-=?-function (element element -> boolean))
%   (sized-search-tree-label-<?-function (element element -> boolean))
%   (sized-search-tree-tree (sized-tree-of false element)))

% (define-record (list-set-of element)
%   make-list-set
%   list-set?
%   (element-=?-function (element element -> boolean))
%   (list-set-list (list-of element)))
% \end{lstlisting}
% %
% Wie immer bei der Abstraktion schauen wir darauf, was die Definitionen
% gemeinsam haben und wo Unterschiede sind:
% %
% \begin{itemize}
% \item Alle Definitionen enthalten eine Funktion, um zwei mögliche
%   Elemente auf Gleichheit zu testen.
% \item Die ersten beiden Definitionen enthalten außerdem noch eine
%   Funktion, um zwei mögliche Elemente zu ordnen.
% \item Alle Definitionen enthalten ein Feld, in dem die eigentliche
%   Datenstruktur mit den Elementen enthalten ist~-- die
%   \textit{Repräsentation} der Menge.  Allerdings ist die Signatur
%   dieses Felds jedesmal eine andere.
% \end{itemize}
% %
% Unterschiedlich sind die \lstinline{member?}- und
% \lstinline{insert}-Funktionen, die jeweils auf die Repräsentation und
% die Gleichheits- und die Ordnungs-Funktion zugreifen.  Die Funktionen
% stehen aber gar nicht direkt in der Record-Definition, greifen aber
% auf die Funktionen für Gleichheit und Ordnung zu.

% Die Lage ist also etwas unübersichtlich, und hier kommt die
% Datenabstraktion ins Spiel.  Sie besteht aus zwei Techniken:
% %
% \begin{itemize}
% \item Wir "<verstecken"> die Signatur der Repräsentation, damit unser
%   Programm nicht geändert werden muss, wenn die Implementierung und
%   damit auch diese Signatur geändert wird.
% \item Wir stecken die \lstinline{member?}- und
%   \lstinline{insert}-Funktionen selbst in den Record.  Im Kontext der
%   Datenabstraktion heißen diese Funktionen
%   \textit{Methoden}.\index{Methode}
% \end{itemize}
% % 
% Daraus wird folgende Daten-Definition mit zugehöriger
% Record-Definition:
% %
% \begin{lstlisting}
% ; Eine Menge besteht aus:
% ; - Repräsentation
% ; - Methode, um Elemente einzufügen
% ; - Methode, um festzustellen, ob Wert Element in Menge ist
% (define-record (set-of element)
%   make-set
%   set?
%   (set-representation any)
%   (set-insert-method (element (set-of element)-> (set-of element)))
%   (set-member?-method (element (set-of element)-> boolean)))
% \end{lstlisting}
% %
% Das Feld \lstinline{set-representation} wird später die
% Repräsentation, also den Suchbaum beziehungsweise die Liste enthalten.
% Die Signatur der Repräsentation hängt davon ab, welche Implementierung
% zum Suchbaum gehört. Die wissen wir an dieser Stelle allerdings nicht.
% Wir wollen außerdem keinen zusätzlichen Parameter zu
% \lstinline{set-of} hinzufügen, weil sie die Signatur der
% Repräsentation sichtbar machen für Funktionen sichtbar machen würde,
% die mit \lstinline{set-of} hantieren.  (Bei der Variante mit
% gemischten Daten wurde ja so ein Parameter auch nicht benötigt.)

% Es liegt nahe, die Gleichheits-Funktion, die alle drei
% Record-Definitionen gemeinsam haben, auch in die Definition von
% \lstinline{set-of} aufzunehmen.  Wir werden aber sehen, dass dies
% unnötig ist.

% Stellen wir also mal eine Menge her.  Der Einfachheit halber benutzen
% wir wieder Listen als Repräsentation.  Hier sind Kurzbeschreibung,
% Signatur und Schablone:
% %
% \begin{lstlisting}
% ; Menge mit Listenrepräsentation erzeugen
% (: make-list-set ((%a %a -> boolean) (list-of %a) -> (set-of %a)))

% (define make-list-set
%   (lambda (element-=? list)
%     (make-set ... ... ...)))
% \end{lstlisting}
% %
% Die erste Eingabe von \lstinline{make-set} ist die Repräsentation, in
% diesem Fall also einfach \lstinline{list}.  Wir benötigen noch die
% beiden Methoden für \lstinline{insert} und \lstinline{member?}.  Die
% schreiben wir als lokale Definitionen und passen sie aus den
% Definitionen des ersten Abschnitts an:
% %
% \begin{lstlisting}
% (define make-list-set
%   (lambda (element-=? list)
%     (define list-set-insert
%       (lambda (value set)
%         (if (list-set-member? value set)
%             set
%             (make-set (cons value list)
%                       list-set-insert list-set-member?))))
%     (define list-set-member?
%       (lambda (value set)
%         (member? element-=?
%                  value
%                  (set-representation set))))
        
%     (make-set list list-set-insert list-set-member?)))
% \end{lstlisting}
% %
% Das Testen der Funktion holen wir nach, weil wir dafür die Funktionen
% \lstinline{set-insert} und \lstinline{set-member?} brauchen.  Die
% kommen gleich.

% Hier ist ein Beispiel für einen Aufruf von \lstinline{make-list-set}:
% %
% \begin{lstlisting}
% ; Menge mit Elementen 1 2 3 4
% (define list-set1
%   (make-list-set = (list 1 2 3 4)))
% \end{lstlisting}
% %
% Kommen wir nun zu \lstinline{set-insert} und \lstinline{set-member?}.
% Die rufen die entsprechenden Methoden auf:
% %
% \begin{lstlisting}
% ; Element in Menge einfügen
% (: set-insert (%a (set-of %a) -> (set-of %a)))

% (define set-insert
%   (lambda (value set)
%     ((set-insert-method set) value set)))

% ; Feststellen, ob Wert Element einer Menge ist
% (: set-member? (%a (set-of %a) -> boolean))

% (define set-member?
%   (lambda (value set)
%     ((set-member?-method set) value set)))
% \end{lstlisting}
% %
% Hier ist ein Beispiel für einen Aufruf von \lstinline{set-insert}:
% %
% \begin{lstlisting}
% ; Menge mit Elementen 1 2 3 4 5
% (define list-set2
%   (set-insert 5 list-set1))
% \end{lstlisting}
% %
% Damit können wir auch \lstinline{make-list-set} rudimentär testen:
% %
% \begin{lstlisting}
% (check-expect (set-member? 2 list-set1) #t)
% (check-expect (set-member? 5 list-set1) #f)
% (check-expect (set-member? 5 list-set2) #t)
% \end{lstlisting}

% \section{Repräsentationswechsel}

% TBD

%%% Local Variables: 
%%% mode: latex
%%% TeX-master: "i1"
%%% End: 

